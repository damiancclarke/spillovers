\documentclass[10pt,letterpaper,subeqn]{beamer}
\setbeamertemplate{navigation symbols}{}
\usefonttheme{serif}
\usecolortheme{seahorse}


\usepackage{amsmath}
\usepackage{amsfonts}
\usepackage{amssymb}
\usepackage[english]{babel}
\selectlanguage{english}
\usepackage{bm}
\usepackage{booktabs}
\usepackage{color}
\usepackage[update,prepend]{epstopdf}
\usepackage{eqnarray}
\usepackage{framed}
\usepackage{fleqn}
\usepackage{graphics}
\usepackage{hyperref}
\usepackage[utf8]{inputenc}
\usepackage{setspace}
\usepackage{textcomp}
\usepackage{wrapfig}
\usepackage{multirow}
\usepackage{caption}
\usepackage{subcaption}
\usepackage{subfloat}
\setbeamertemplate{caption}[numbered]
\usepackage{wrapfig}
\usepackage{tikz}

\definecolor{cadmiumgreen}{rgb}{0.0, 0.42, 0.24}


%================================================================================
%== TITLE, NAMES, DATE
%================================================================================
\title{Estimating Difference-in-Differences in the Presence of Spillovers}
%\subtitle{Theory and Application to Contraceptive Reforms in Latin America}
\author{Damian Clarke\inst{\dag} }
\institute{\inst{\dag}  University of Oxford}
\date{August 2015}


%================================================================================
%== Document
%================================================================================
\begin{document}


\begin{frame}
\titlepage
\end{frame}
%================================================================================

\section{Introduction}
\begin{frame}[label=int1]
  \frametitle{This Paper}

\end{frame}

\section{Introduction}
\begin{frame}[label=int2]
  \frametitle{This Paper}

\end{frame}



%================================================================================
\section{Motivation}
\begin{frame}[label=motivation]
  \frametitle{Motivation}

\end{frame}

\begin{frame}[label=DDM]
  \frametitle{Difference-in-differences in Economics}
\begin{itemize}
 \item 
 \item 
 \item Consistency relies on the Stable Unit Treated Value Assumption (SUTVA)
 \item Much recent discussion within and outside of academics (eg ``The Worm Wars'')
\end{itemize}
\end{frame}

\begin{frame}[label=teenPreg]
  \frametitle{Teenage Pregnancy in Latin America}

\end{frame}



%================================================================================
\section{Methodology}
\begin{frame}[label=method1]
  \frametitle{Methodology}
``Typical'' diff-in-diff where outcome is generated by components of variance 
process (Ashenfelter and Card, 1985):
\vspace{4mm}
\begin{equation}
\label{Seqn:COV}
Y(i,t)=\delta(t) + \alpha D(i,t)+\eta(i)+\nu(i,t),
\end{equation}
\vspace{4mm}
and estimand of interest is:
\vspace{4mm}
\begin{eqnarray}
\label{Seqn:DD}
\alpha&=\{E[Y(i,1)|D(i,1)=1]-E[Y(i,1)|D(i,1)=0]\} \\ \nonumber
      &-\{E[Y(i,0)|D(i,1)=1]-E[Y(i,0)|D(i,1)=0]\}
\end{eqnarray}
\end{frame}

\begin{frame}[label=method2]
  \frametitle{Methodology}
This relies on a binary measure of treated versus non-treated.  In this paper,
I generalise this.  Consider:
\vspace{5mm}
\begin{equation}
\label{Seqn:COV2}
Y(i,t)=\delta(t) + \alpha D(i,t)+\textcolor{red}{\beta R(i,t)}+\eta(i)+\nu(i,t)
\end{equation}
\vspace{3mm}
where
\vspace{3mm}
\[
 R(i,t) =
  \begin{cases}
   f\Big(X(i,t)\Big)>0   & \text{if close to, but not in, treatment area} \\ 
   0                            & \text{otherwise} 
  \end{cases}
\]

\end{frame}

\begin{frame}[label=method3]
  \frametitle{Methodology}
\begin{itemize}
\item Here $X(i,t)$ is an observed measure of `distance'
\item And $f(\cdot)$ is a positive monotone function
\item However, $R(i,t)$ is not observed given that ``close'' is subjective
\end{itemize}
\end{frame}


\begin{frame}[label=method4]
  \frametitle{Methodology}
And this leads to two estimands:
\begin{eqnarray}
\nonumber
\label{Seqn:DDa}
\alpha&=\{E[Y(1)|D(1)=1,R(1)=0]-E[Y(1)|D(1)=0,R(1)=0]\} \\ \nonumber
      &\ -\ \{E[Y(0)|D(1)=1,R(1)=0]-E[Y(0)|D(1)=0,R(1)=0]\}, 
\end{eqnarray}

\begin{eqnarray}
\nonumber
\beta&=\{E[Y(1)|D(1)=0,R(1)\neq 0]-E[Y(1)|D(1)=0,R(1)=0]\} \\ \nonumber
      &\ -\ \{E[Y(0)|D(1)=0,R(1)\neq 0]-E[Y(0)|D(1)=0,R(1)=0]\}. 
\end{eqnarray}
\vspace{3mm} \\
where $i$ has been supressed for ease of exposition.
\end{frame}

%================================================================================
\section{Estimation}
\begin{frame}[label=estim]
  \frametitle{Estimation}
In this paper I am interested in estimating the unbiased causal effects $\alpha$ 
and $\beta$.  This is referred to as a ``spillover robust double differences
estimator''
\vspace{4mm}
\begin{itemize}
\item This allows for spillovers, without imposing that they must exist
\item More importantly, if spillovers do exist, this corrects for potential 
confounding effects of including ``close'' units in the control group
\item 
\end{itemize}

\end{frame}



\section{Empirical Application}
\begin{frame}[label=empir]
  \frametitle{Empirical Application}
Hence, I define:
\vspace{4mm}
\begin{eqnarray}
\label{Seqn:estimATT}
ATT=E[Y^1(i,1)-Y^0(i,1)|D(i,1)=1]\  \\
\label{Seqn:estimATC}
ATC=E[Y^1(i,1)-Y^0(i,1)|C(i,1)=1],
\end{eqnarray}
\vspace{4mm} \\
These are the sample counterparts of $\alpha$ and $\beta$, the population 
estimands of interest.
\vspace{8mm} \\
\begin{enumerate}
\item[ATT] Average treatment effect on the treated
\item[ATC] Average treatment effect on the close to treated
\end{enumerate}
\end{frame}


\subsection{Empirical Application 1: Emergency Contraception in Chile}
\begin{frame}[label=empirA]
  \frametitle{Empirical Application 1: Emergency Contraceptive Pill in Chile} 

\end{frame}


\subsection{Empirical Application 2: Legal Interuption of Pregnancy in Mexico DF}
\begin{frame}[label=empirB]
  \frametitle{Empirical Application 2: Legal Interuption of Pregnancy in Mexico DF}

\end{frame}



%================================================================================
\section{Conclusion and Future Work}
\begin{frame}[label=concl]
  \frametitle{Conclusion and Future Work}

\end{frame}








\end{document}

