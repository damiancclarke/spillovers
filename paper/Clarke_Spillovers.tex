%********************************************************************************
\documentclass[12pt]{article}

\usepackage[margin=0.84in]{geometry} 
\usepackage{appendix}
\usepackage{natbib}
\bibliographystyle{abbrvnat}
\bibpunct{(}{)}{;}{a}{,}{,}
\usepackage{setspace}
\usepackage{amsmath}
\usepackage{amsfonts}
\usepackage{amsthm}
\usepackage{amssymb}
%\newtheorem{assumption}{Assumption}
\newtheorem{proposition}{Proposition}
\usepackage{booktabs}
\usepackage{bm}
\usepackage[usenames, dvipsnames]{color}
\usepackage{epsfig}
\usepackage{epstopdf}
\usepackage{graphics}
\usepackage{hyperref}
\usepackage{pdflscape}
\usepackage[capposition=top]{floatrow}
\usepackage{subcaption}
\usepackage{subfloat}
\usepackage{dsfont}
%\usepackage{newcent}
%\usepackage{millennial}
%\usepackage{mathastext}
\usepackage{libertine}
\usepackage[libertine]{newtxmath}
\usepackage{wrapfig}

\DeclareMathOperator*{\argmin}{arg\,min}


\newcommand{\sdidloc}{./../../spillovers}
\newcommand{\JELs}{\noindent  \\}

\hypersetup{
    colorlinks=true,      
    linkcolor=BlueViolet, 
    citecolor=BlueViolet, 
    filecolor=BlueViolet, 
    urlcolor=BlueViolet   
}

\setlength\parskip{0.12in}
%\setlength\topmargin{-0.375in}
%\setlength\textheight{8.8in}
%\setlength\textwidth{5.8in}
%\setlength\oddsidemargin{0.4in}
%\setlength\evensidemargin{-0.5in}
%

\newtheorem*{assumption*}{\assumptionnumber}
\providecommand{\assumptionnumber}{}
\makeatletter
\newenvironment{assumption}[2]
 {%
  \renewcommand{\assumptionnumber}{Assumption #1{#2}}%
  \begin{assumption*}%
  \protected@edef\@currentlabel{#1}%
 }
 {%
  \end{assumption*}
 }
\makeatother
\newcommand{\asref}[2]{\ref{#1}{\textcolor{BlueViolet}{#2}}}


\newcommand{\Var}{\mathrm{Var}}
\newcommand{\Cov}{\mathrm{Cov}}
\newcommand{\Bias}[2]{\frac{\Cov[#1,#2]}{\Var[#1]}}


%ABSTRACT WITHIN CHAPTER
\usepackage{changepage}%
\makeatletter
\newenvironment{chapabstract}{%
    \begin{center}%
      \bfseries Abstract
    \end{center}}%
   {\par}
\makeatother
\newif\ifheading



%\title{\textbf{Estimating Difference-in-Differences in the Presence of Spillovers: 
%Theory and Application to Contraceptive Reforms in Latin America}
\title{Estimating Difference-in-Differences in the Presence of Spillovers%
  \thanks{I thank Paul Devereux, James Fenske, Rossa Keeffe-O'Donovan, Rudi Rocha,
    Chris Roth and Margaret
    Stevens for comments and suggestions which have improved this paper. I am also
    grateful to audiences at PUC Chile, Universidad de la Rep\'ublica Uruguay, and
    participants in the Impact Evaluation Meeting at the Inter-American Development
    Bank for their comments. I gratefully acknowledge the financial support of
    FONDECYT grant (number 11160200) provided by the Government of Chile.  Any
    remaning errors are my own. Full source code, including the Stata module cdifdif
    is available for download and use at 
\href{https://github.com/damiancclarke/spillovers}{https://github.com/damiancclarke/spillovers}.  
Affiliation: Department of Economics, Universidad de Santiago de Chile. Contact email:
\href{mailto:damian.clarke@usach.cl}{damian.clarke@usach.cl}.}}
\author{Damian Clarke}
\date{\today}

\begin{document}
%\fontfamily{ppl}
\maketitle
\begin{spacing}{1.3}
\begin{abstract}
  I propose a method for difference-in-differences (DD) estimation in
  situations where the stable unit treatment value assumption is violated
  locally. This is relevant for a wide variety of cases where spillovers may
  occur between quasi-treatment and quasi-control areas in a (natural)
  experiment. A flexible methodology is described to test for such spillovers,
  and to consistently estimate treatment effects in their presence. This
  spillover-robust DD method results in two classes of estimands: treatment
  effects, and ``close'' to treatment effects. The methodology outlined
  describes a versatile and non-arbitrary procedure to determine the distance
  over which treatments propogate, where distance can be defined in many ways,
  including using multi-dimensional measures. This methodology is illustrated
  by simulation, and by its application to estimates of the impact of
  state-level text-messaging bans on fatal vehicle accidents.  Extending
  existing DD estimates, I document that reforms travel over roads, and have
  spillover effects in neighbouring non-affected counties.
\end{abstract}
\noindent\emph{JEL codes}: C13, C21, D04, J13, R23. \\
\noindent\emph{Keywords}: Policy evaluation, difference-in-differences,
spillovers, natural experiments, SUTVA
\newpage
%================================================================================
\section{Introduction}
Natural experiments often rely on territorial borders to estimate treatment 
effects.  These borders separate quasi-treatment from quasi-control groups with
individuals in one area having access to a program or treatment while those in 
another do not.  In cases such as these where geographic location is used to 
motivate identification, the stable unit treatment value assumption (SUTVA) is, 
either explicitly or implicitly, invoked.\footnote{The SUTVA has a long and 
interesting history, under various guises. \citet{Cox1958} refers to ``no 
interference between different units'', before \citet{Rubin1978} introduced the 
concept of SUTVA (the name SUTVA did not appear until \citet{Rubin1980}).  
Recent work of \citet{Manski2013}, refers to this assumption as Individualistic 
Treatment Response (ITR). I provide additional discussion of related
literarature in Appendix \ref{app:literature} of this paper.}

However, often territorial borders are porous.  Generally state, regional,
municipal, and village boundaries can be easily, if not costlessly, crossed.
Given this, researchers interested in using natural experiments in this way may
be concerned that the effects of a program in a treatment cluster may spillover 
into non-treatment clusters---at least locally.

Such a situation is in clear violation of the SUTVA's requirement that the 
treatment status of any one unit must not affect the outcomes of any other unit.  
In this paper I propose a methodology to deal with such spillover effects.  I
discuss how to test for local spillovers, and if such spillovers exist, how to 
estimate unbiased treatment effects in their presence.  It is shown that this 
estimation requires a weaker condition than SUTVA: namely that SUTVA holds 
between \emph{some} units, as determined by their distance from the treatment 
cluster.  I discuss how to estimate treatment and spillover effects, and then
propose a method to generalise the proposed estimator to a higher dimensional 
case where spillovers may depend in a flexible way on an arbitrary number of 
factors.

It is shown that this methodology recovers unbiased treatment estimates under 
quite general violations of SUTVA.  While it is assumed that the distance of 
an individual to the nearest treatment cluster determines whether stable unit 
treatment type assumptions hold for that individual, `distance' is defined 
very broadly.  It is envisioned that this will allow for phenomena such as 
information flowing from treated to untreated areas, or of untreated 
individuals violating their treatment status by travelling from untreated to
treated areas.  In each case distance plays a clear role in the propogation 
of treatment; either information must travel out, or beneficiaries must travel 
in. Similarly, this framework allows for local general equilibrium-type 
spillovers, where a tightly applied program may have an economic effect on 
nearby markets, but where this effect disipates as distance to treatment 
increases.

This methodology has two particular features that make it suitable for
application to empirical work.\footnote{Furthermore, computer programs are made
  available by the author to implement the described estimation procedure in
  a number of statistical languages.}  Firstly, it places no strict restriction
on the way in which spillovers propogate between observations \emph{and}
between treatment clusters.  A range of other methods of estimating indirect
policy effects have been proposed which are based on a hierarchical
treatment assignment, where treatment receipt is allowed to occur within a
particular geographic cluster, but not to neighbouring clusters (see for example
\citet{HudgensHalloran2008,LiuHudgens2014,Bairdetal2014} for some such cases).
However, the spillover-robust DD method laid out here allows spillovers
of treatments from treated clusters to non-treated clusters, with the only
restriction being a geographical dependence of propagation.  Secondly,
spillover distances are determined in an optimising (non ad-hoc) way.  A
Root Mean Squared optimal method is proposed to determine the size of distance
bins to be considered, with some similarities to bandwidth search in regression
discontiunity models.  This optimising procedure provides a simple automated
rule to determine spillover distances, which removes any parameter choices from
a researcher's control, allowing for the avoidance of concerns that parameters
may have been chosen in order to support a particular hypothesis.  This
procedure allows for spillovers to be determined endogenously from data.
A data-snooping procedure is described, along with refinements for use with
large datasets.  This described procedure is well-suited to
difference-in-difference applications which previously have based the
estimation of externalities or geographic spillovers on researcher-defined
distance cut-offs (a number of important empirical examples of this type
include \citet{MiguelKremer2004,Almondetal2009}).

The performance of the proposed estimator is examined, firstly, by simulation,
and secondly by application to a particular empirical example.  Under
simulation I show that the proposed estimator recovers estimates of the
treatment effect of interest, and has good size properties, even in cases
where spillovers occur to a large proportion of control units.  The estimator
is documented to perform well, even under model mis-specification of the
spillover bins, given the flexible modeling procedure employed.

In turning to empirics, this methodology is illustrated by considering the
case of the roll-out of state-level text messaging bans for vehicle operators
in the US.  We return to the data and specifications of
\citet{AboukAdams2013}, who document the impacts of these text-messaging
bans on fatal vehicle accidents using single-vehicle single-occupant
accidents, due to the increased likelihood that these accidents owe to
the use of mobile telephones.  We revisit their estimates using the precise
geographic location of each accident, and county level figures for the US.
Following their specifications, we find that allowing for spillovers suggests
that counties which were not directly treated by the reform but which are
located close to treated areas are impacted in a similar way as those
which were directly treated.  This is a relevant result for policy evaluation,
as it suggests that the original reforms may have wider impacts than originally
determined, and, importantly, that drivers did not simply delay the sending
of text messages until they were travelling on roads in nearby areas without
text messaging bans.  The optimal spillover procedure finds, however, that
changes in driver behavour are perceptible over relatively short distances,
of anywhere from 0-18 kilometres, depending on the reform type examined.

Although the empirical example examined uses a geographic measure of distance,
this methodology should not be considered as limited to only spatial spillovers.
Univariate measures of distance including propogation through nodes in a
network, ethnic distance, ideological distance, or other quantifiable measures
of difference between units can be used in precisely the same manner using the
results and techniques described in this paper.  I also show how multivariate
measures of distance, or interactions between distance and other variables, can
be similarly employed.  This is particularly useful for cases where the effects
of spillovers may be expected to vary by individual characteristics such as age,
socioeconomic status, access to transport or access to information.

This paper joins recent literature which aims to loosen the strong structure 
imposed by the SUTVA.  Perhaps most notably, it is (in broad terms) an 
application of \citeauthor{Manski2013}'s (2013) social interactions framework, 
and \citet{AronowSamii2017}'s general interference framework, focusing on the
case where spillovers are restricted to areas local to treatment clusters.
However, as discussed above, unlike recent developments focusing on spillovers 
between treated and control units \emph{within} a treatment cluster (notable
examples in the economics literature include \citet{McIntosh2008,Bairdetal2014,
  AngelucciDeGiorgi2009,AngelucciDiMaro2010}), this paper focuses on situations where entire clusters
are treated, and the status of the \emph{cluster} may affect nearby non-treated
clusters.  This is likely the case for quasi-experimental studies common in DD
models, where `experiments' are defined based on geographic boundaries, such as
administrative political regions which set different policies.\footnote{A very
  different case is that of (for example) PROGRESA/Oportunidades, where treatment
  clusters (ie localities or \emph{localidades}) contained both treatment and
  control individuals, and the literature is concerned with spillovers between
  treatment and control individuals within this treatment cluster.}

While being of direct relevance for the estimation of both treatment and
spillover tests in a difference-in-difference setting, the spillover-robust
DD procedure described in this paper is also a generally useful specification
test which can be applied by authors wishing to partially test the assumptions
underlying DD estimates.  Empirical papers using DD estimates often estimate
event-study specifications as a way to examine whether dependence over time
is observed in changes between treatment and control areas around the reform
date.  The tests outlined in this paper provide a similar specification test,
however rather than considering temporal spillovers holding geography constant,
we consider spatial spillovers holding time constant.  Thus, as event studies
can be considered as partial tests of the parallel trend assumption in
difference-in-differences, the spillover-robust DD model can be considered as a
partial test of the SUTVA, both of which underly the unbiasedness of DD
estimates. The parallels between event studies and spillover-robust DD estimates
are also drawn in that both can be considered necessary, but not sufficient to
motivate the unbiased estimation of DD models.


\nocite{Heckmanetal1998}

%================================================================================
\section{Methodology}
Define $Y(i,t)$ as the outcome for individual $i$ and time $t$.  The population
of interest is observed at two time periods, $t\in \{0,1\}$. Assume that between
$t=0$ and $t=1$, some fraction of the population is exposed to a 
quasi-experimental treatment.  As per \citet{Abadie2005}, I will denote 
treatment for individual $i$ in time $t$ as $D(i,t)$, where $D(i,1)=1$ implies 
that the individual was treated, and $D(i,1)=0$ implies that the individual was
not directly treated.  Given that treatment only exists between periods 0 and 1,
$D(i,0)=0\ \forall\ i$.

It is shown by \citet{AshenfelterCard1985} that if the outcome is generated by
a component of variance process:
\begin{equation}
\label{Seqn:COV}
Y(i,t)=\delta(t) + \alpha D(i,t)+\eta(i)+\nu(i,t)
\end{equation}
where $\delta(t)$ refers to a time-specific component, $\alpha$ as the impact of 
treatment, $\eta(i)$ a component specific to each individual, and $\nu(i,t)$ as 
a time-varying individual (mean zero) shock, then a sufficient condition for 
identification (a complete derivation is provided by \citet{Abadie2005}) is:
\begin{equation}
\label{Seqn:ID}
P(D(i,1)=1|\nu(i,t))=P(D(i,1)=1) \ \forall\ t\in\{0,1\}.
\end{equation}
In other words, identification requires that selection into treatment does not
rely on the unobserved time-varying component $\nu(i,t)$.  If this condition 
holds, then the classical DD estimator provides an unbiased estimate of the
treatment effect:
\begin{equation}
\label{Seqn:DD}
\begin{split}
\alpha&=\{E[Y(i,1)|D(i,1)=1]-E[Y(i,1)|D(i,1)=0]\} \\
      &-\{E[Y(i,0)|D(i,1)=1]-E[Y(i,0)|D(i,1)=0]\}
\end{split}
\end{equation}
where $E$ is the expectations operator.

Assume now, however, that treatment is not precisely geographically bounded.  
Specifically, I propose that those living in control areas `close to' treatment 
areas are able to access treatment, either partially or completely.  Such a 
case allows for a situation where individuals `defy' their treatment status, by 
travelling or moving to treated areas, or where spillovers from treatment 
areas is diffused through general equilibrium processes.  Define $R(i,t)$ 
where:
\begin{equation}
\nonumber
 R(i,t) =
  \begin{cases}
   f\Big(X(i,t)\Big)>0   & \text{if an individual resides close to, but not in, a treatment area} \\
   0                            & \text{otherwise} 
  \end{cases}
\end{equation}
Where $X(i,t)$ is an individual covariate measuring distance (in a very general 
sense) to treatment and $f(\cdot)$ is a positive monotone function. As treatment 
occurs only in 
period 1, $R(i,0)=0$ for all $i$.  Similarly, as living in a treatment area 
itself excludes individuals from living `close to' the same treament area, 
$R(i,t)=0$ for all $i$ such that $D(i,t)=1$.

Generalising from (\ref{Seqn:COV}), now I assume that $Y(i,t)$ is generated 
by:
\begin{equation}
\label{Seqn:COV2}
Y(i,t)=\delta(t) + \alpha D(i,t)+\beta R(i,t)+\eta(i)+\nu(i,t)
\end{equation}
If we observe only $Y(i,t)$, $D(i,t)$ and $R(i,t)$, a sufficient condition for 
estimation now consists of (\ref{Seqn:ID}) and the following assumption: 
\begin{equation}
\label{Seqn:ID2}
P(R(i,1)\neq 0|\nu(i,t))=P(R(i,1)\neq 0) \ \forall\ t\in\{0,1\}.
\end{equation}
This requires that both treatment, and being close to treatment cannot depend 
upon individual-specific time-variant components. To see this, write 
(\ref{Seqn:COV2}), adding and subtracting the individual-specific component
$E[\eta(i)|D(i,1),R(i,1)]$:
\begin{equation}
\label{Seqn:addsub}
Y(i,t)=\delta(t) + \alpha D(i,t)+\beta R(i,t)+E[\eta(i)|D(i,1),R(i,1)]+\varepsilon(i,t)
\end{equation}
where, following \citet{Abadie2005}, $\varepsilon(i,t)=\eta(i)-E[\eta(i)|D(i,1),R(i,1)]
+\nu(i,t)$.  We can write $\delta(t)=\delta(0)+[\delta(1)-\delta(0)]t$, and write
$E[\eta(i)|D(i,1),R(i,1)]$ as the sum of the expectation of the individual-specific 
component $\eta(i)$ over treatment status and `close' status\footnote{$E[\eta(i)|
D(i,1),R(i,1)]=E[\eta(i)|D(i,1)=0,R(i,1)=0]+(E[\eta(i)|D(i,1)=1,
R(i,1)=0]-E[\eta(i)|D(i,1)=0,R(i,1)=0])\cdot D(i,1)+(E[\eta(i)|D(i,1)=0,R(i,1)\neq 0]-
E[\eta(i)|D(i,1)=0,R(i,1)=0])\cdot R(i,1)$.}.  Finally define $\mu$ (the intercept at
time 0) as:
\[
\mu=E[\eta(i)|D(i,1)=0,R(i,1)=0]+\delta_0,
\]
$\tau$, a fixed effect for treated individuals, as 
\[
\tau=E[\eta(i)|D(i,1)=1,R(i,1)=0]-E[\eta(i)|D(i,1)=0,R(i,1)=0], 
\]
$\gamma$, a similar fixed effect for individuals close to treatment, as 
\[
\gamma=E[\eta(i)|D(i,1)=0,R(i,1)\neq 0]-E[\eta(i)|D(i,1)=0,R(i,1)=0]
\] and $\delta$, a time trend, as $\delta=\delta(1)-\delta(0)$.  Then 
from the above and (\ref{Seqn:addsub}) we have:
\begin{equation}
\label{Seqn:cDD}
Y(i,t)=\mu+\tau D(i,1) + \gamma R(i,1) + \delta t + \alpha D(i,t) + \beta R(i,t) + 
       \varepsilon(i,t).
\end{equation}
Notice that this (estimable) equation now includes the typical DD fixed effects 
$\tau$ and $\delta$ and the double difference term $\alpha$.  However it also 
includes `close' analogues $\gamma$ (an initial fixed effect), and $\beta$: the 
effect of being `close to' a treatment area.

From the assumptions in (\ref{Seqn:ID}) and (\ref{Seqn:ID2}) it holds that 
$E[(1,D(i,1),R(i,1),D(i,t),$
$R(i,t))\cdot\varepsilon(i,t)]=0$, which 
implies that all parameters from (\ref{Seqn:cDD}) are consistently estimable 
by OLS.  Importantly, this includes consistent estimates of $\alpha$ and 
$\beta$: the effect of the program treatment and spillover effects on 
outcome variable $Y(i,t)$.  Then, from (\ref{Seqn:cDD}), a our coefficients 
of interest $\alpha$ and $\beta$ are:
\begin{equation}
\nonumber
\label{Seqn:DDa}
\begin{split}
\alpha&=\{E[Y(i,1)|D(i,1)=1,R(i,1)=0]-E[Y(i,1)|D(i,1)=0,R(i,1)=0]\} \\
      &-\{E[Y(i,0)|D(i,1)=1,R(i,1)=0]-E[Y(i,0)|D(i,1)=0,R(i,1)=0]\}, 
\end{split}
\end{equation}
and 
\begin{equation}
\nonumber
\label{Seqn:DDb}
\begin{split}
\beta&=\{E[Y(i,1)|D(i,1)=0,R(i,1)\neq 0]-E[Y(i,1)|D(i,1)=0,R(i,1)=0]\} \\
      &-\{E[Y(i,0)|D(i,1)=0,R(i,1)\neq 0]-E[Y(i,0)|D(i,1)=0,R(i,1)=0]\}. 
\end{split}
\end{equation}
where the sample estimate of each parameter is generated by a least squares
regression of (\ref{Seqn:cDD}) using a random sample of 
$\{Y(i,t), D(i,t), R(i,t): i=1, \ldots, N, t=0, 1\}$.

%================================================================================
\section{A Spillover-Robust Double Differences Estimator}
\label{Sscn:estim}
We are interested in estimating difference-in-difference parameters $\alpha$ and 
$\beta$ from (\ref{Seqn:cDD}).  I will refer to these estimators respectively
as the average treatment effect on the treated (ATT), and the average treatment
effect on the close to treated (ATC).  Average treatment effects are cast in 
terms of the \citet{Rubin1974} Causal Model.

Following a potential outcome framework, I denote $Y^1(i,t)$ as the potential
outcome for some person $i$ at time $t$ if they were to receive treatment, and
$Y^0(i,t)$ if the person were not to receive treatment.  Our ATT and ATC are
thus:
\begin{eqnarray}
\label{Seqn:estimATT}
ATT=E[Y^1(i,1)-Y^0(i,1)|D(i,1)=1]\  \\
\label{Seqn:estimATC}
ATC=E[Y^1(i,1)-Y^0(i,1)|C(i,1)=1],
\end{eqnarray}
where I define a new binary variable $C(i,t)$, which indicates if individuals 
are close or not close to treatment.  This is simply a redefinition of $R(i,t)$,
where $C(i,t)=\mathbf{1}_{R(i,t)\neq 0}$.  Given that for now we are interested
in the \emph{average} effect on those close to treatment we condition only on
$C(i,t)$, however in the sections which follow extend to a more general form of
$R(i,t)$ to examine the rate of decay or propogations of spillovers over space.

As is typical in the potential outcome literature, estimation is hindered by the
reality that only one of $Y^1(i,t)$ or $Y^0(i,t)$ is observed for a given 
individual $i$ at time $t$.  The realised outcome can thus be expressed as 
$Y(i,t)=Y^0(i,t)\cdot (1-D(i,t))(1-C(i,t))+Y^1(i,t)\cdot D(i,t)+Y^1(i,t)\cdot 
C(i,t)$, where, depending on an individual's time varying treatment and close
status, we observe either $Y^0(i,t)$ (untreated) or $Y^1(i,t)$ (treated or 
close).  Thus, in order to be able to estimate the quantities of interest, we 
rely on averages over the entire population, rather than average of individual 
treatment effects.  As is typical in difference-in-differences identification
strategies, consistent estimation requires parallel trends assumptions.  In the 
case of treatment \emph{and} local spillovers, this relies on:

\begin{assumption}{1}{}
\label{Sass:PT}
\textbf{Parallel trends in treatment and control:} \\
$E[Y^0(i,1)-Y^0(i,0)|D(i,1)=1,C(i,1)=0]=
E[Y^0(i,1)-Y^0(i,0)|D(i,1)=0,C(i,1)=0]$,
\end{assumption}
\begin{assumption}{2}{}
\label{Sass:PTC}
\textbf{Parallel trends in close and control:} \\
$E[Y^0(i,1)-Y^0(i,0)|D(i,1)=0,C(i,1)=1]=
E[Y^0(i,1)-Y^0(i,0)|D(i,1)=0,C(i,1)=0]$.
\end{assumption}

In other words, assumption \ref{Sass:PT} and \ref{Sass:PTC} state that in the 
absence of treatment, the evolution of outcomes for treated units and for units 
close to treatment would have been parallel to the evolution of entirely 
untreated units.  This is the fundamental DD identifying assumption of parallel 
trends, generalised to hold for treatment \emph{and} close to treatment status.  
Note that in the above, we no longer need to make \emph{any} assumptions 
regarding parallel trends between treatment and close to treatment units 
allowing for direct interactions between those living in treatment areas, and
those living close by.

However, as a matter of course, in order to consistently estimate any treatment 
effect, some form of the SUTVA must be invoked.  Typically, this requires that 
each individual's treatment status does not affect each other individual's 
potential outcome.  Here, I loosen SUTVA. In the remainder of this article, it 
will be assumed that:
\begin{assumption}{3}{}
\label{Sass:SUTVAs}
\textbf{SUTVA holds for some units:} \\
There is some subset of individuals $j\in J$ of the total population $i\in N$ 
for whom potential outcomes ($Y_j^0, Y_j^1$) are independent of the treatment 
status $D=\{0,1\}\ \forall_{i\neq j} \in N$.
\end{assumption}
\vspace{-4mm}
\noindent Fundamentally, this assumption implies that SUTVA need not hold among 
all units.  Now, rather than identification relying on each unit not affecting 
each other unit, it relies on there existing at least some subset of units which 
are not affected by the treatment status of others.\footnote{This is an 
identifying assumption. If all `non-treatment' units are affected by spillovers 
from the treatment area, a consistent treatment effect cannot be estimated using 
this methodology. This is a general rule and can be couched in 
\citet{HeckmanVytlacil2005}'s terms: `The treatment effect literature 
investigates a class of policies that have partial participation at a point in 
time so there is a ``treatment'' group and a ``comparison'' group. It is not 
helpful in evaluating policies that have universal participation.' (or in this 
case, universal participation and spillovers.}

Finally, I assume that spillovers, or violations of SUTVA, do not occur randomly
in the population:
\begin{assumption}{4}{A}
\label{Sass:SUTVAl}
\textbf{Assignment to close to treatment depends on observable $X(i,t)$:} \\ 
There exists an assignment rule $\delta\Big(X(i,t)\Big)=\{0,1\}$ which maps 
individuals to close to treatment status $C(i,t)$, where $\delta\Big(X(i,t)\Big)=
\mathbf{1}_{X(i,t)<d}$, $X(i,t)$ is an observed covariate, and $d$ is a fixed
%but unknown (though reveal this and how to estimate it later)
scalar cutoff. 
\end{assumption}
\vspace{-4mm}
\noindent This restriction is quite strong, and is loosened in coming sections.  
In other words, it simply states 
that violations of SUTVA occur in an observable way.  For example, if SUTVA does
not hold locally to the treatment area, assumption \asref{Sass:SUTVAl}{A} implies
that we are able to define what `local' is.  While this article focuses on
an $X_i$ representing geographic distance, these derivations do not imply that 
this must be the case.  The `close' indicator $C(i,t)$ could depend on a range 
of phenomena including euclidean space, ethnic distance, edges between
nodes in a network, or, as I return to discuss in section \ref{Ssscn:multi}, 
multi-dimensional interactions between measures such as these and economic 
variables. 
\begin{proposition}
\label{Pass:ATT}
Under assumptions \ref{Sass:PT} to \asref{Sass:SUTVAl}{A}, the ATT and ATC can be 
consistently estimated by least squares when controlling, parametrically or
non-parametrically, for $C(i,t)=\mathbf{1}_{X(i,t)\leq d}$.
\end{proposition}
\noindent Proofs of propositions are offered in appendix \ref{app:proof}. $\qed$

In the following two subsections I examine these estimands in turn. 

%================================================================================
\subsection{Estimating the Treatment Effect in the Presence of Spillovers}
\label{Ssscn:TE}
From proposition \ref{Pass:ATT}, we can consistently estimate $\alpha$ and 
$\beta$, our estimands of interest, with information on treatment status, and 
close to treatment status, along with outcomes $Y(i,t)$ at each point in time. In 
a typical DD framework, we observe $Y(i,t)$ and $D(i,t)$, however, do not fully 
observe $C(i,t)$, an individual's close/non-close status.

We do however, assume that $X(i,t)$, the variable measuring `distance' to 
treatment is observed. From assumption \asref{Sass:SUTVAl}{A}, we could thus map 
$X(i,t)$ to $C(i,t)$ (and later to the heterogeneous function $R(i,t)$) using the 
indicator function, \emph{if} we know the scalar value $d$, which represents the 
threshold of what is considered `close to treatment'. \emph{Ex ante}, in the 
absence some economic model, there is no reason to believe that $d$ will be 
observed by researchers.\footnote{That is not to say that economic intuition 
cannot play a role in suggesting what a reasonable value of $d$ might be. For 
example, if treatment is the receipt of a program with a clear expected value and 
travel costs to access the program increase with distance, there will exist a 
clear cut-off point beyond which individuals will be unwilling to travel. 
Similarly, if treatment must be accessed in a fixed amount of time and 
propogation of treatment is not instantaneous, a limit for $d$ may be calculable. 
This is a point I return to in empirical estimates where one illustration is 
based on access to the emergency contraceptive pill. This point is discussed
in the comprehensive work of social interactions from \citet{Manski2013},
who states that:
  \begin{quote}
    ``response functions are not primitives but rather are
    quantities whose properties stem from the mechanism under study.''
    \citep[p.\ S14]{Manski2013}
  \end{quote}
  In the model laid out here, response functions can be considered as the degree
  that distance from treatment can have an impact on outcomes of interest.
}
In the remainder of this 
section I discuss how to determine $C(i,t)$ based on $X(i,t)$, in the absence of 
a known value for $d$.

In order to do so, we re-write (\ref{Seqn:cDD}) as:
\begin{equation}
\label{Seqn:cDDconc}
\tilde{Y}(i,t)=\mu + \alpha D(i,t) + \upsilon(i,t).
\end{equation}
where $\upsilon(i,t)=\beta R(i,t)+\varepsilon(i,t)$, and for ease of notation
the fixed effects $D(i,1), R(i,1)$ and $t$ have been concentrated out to form
$\tilde{Y}(i,t)$ in line with the  Frisch--Waugh--Lovell (FWL) theorem.  If we 
were to estimate $\hat\alpha$ from the above regression ignoring the potential 
presence of spillovers, then we have that the expectation of $\hat\alpha$ is:
\begin{eqnarray}
\label{Seqn:alphaExp}
E[\hat\alpha] &=& \alpha + \beta\Bias{D(i,t)}{R(i,t)}+\Bias{D(i,t)}{\varepsilon(i,t)} \nonumber \\ 
              &=& \alpha + \beta\Bias{D(i,t)}{R(i,t)},
\end{eqnarray}
where the second line comes from (\ref{Seqn:ID}), which implies that 
$E[\Cov(D(i,t),\varepsilon(i,t))]=0$.  So far we have attached no functional form 
to $R(i,t)$.  Define $R(i,t)$ as:
\begin{equation}
\label{Seqn:Runpack}
R(i,t) = R^1(i,t)+R^2(i,t)+ \cdots + R^K(i,t)
\end{equation}  
where:
\begin{equation}
\label{Seqn:Rpar}
 R^k(i,t) =
  \begin{cases}
   1   & \text{if\ \ } X_i\geq(k-1)\cdot h \text{\ \ and \ } X_i<k\cdot h \\
   0   & \text{otherwise} 
  \end{cases}\ \ \ \ \ \forall k \in (1,2,\ldots,K).
\end{equation}
In the above expression $h$ refers to a bandwidth type parameter, which 
partitions the continuous distance variable $X_i$ into groups of distance $h$.%
\footnote{So, if for example $X_i$ refers to physical distance to treatment and 
the minimum and maximum distances are 0 and 100km respectively, $h$ could be set 
as 5km, resulting in 20 different indicators $R^k$, of which each individual $i$ 
in time $t$ can have at most one switched on.}  Beyond the assumptions made up
to this point, we place no additional limits on how each $R^{k}(i,t)$ variable
is related to the outcome of interest.  We thus allow freely varying paramters
on each $R^{k}(i,t)$, which we will denote $\beta_k$ when included in the DD
model of interest.\footnote{In terms of equation
  \ref{Seqn:cDD}, this implies that the model can now be re-written as:
  \[
  Y(i,t)=\mu+\tau D(i,1) + \gamma_1 R^1(i,1) + \cdots + \gamma_K R^K(i,1) + \delta t
  + \alpha D(i,t) + \beta_1 R^1(i,t) + \cdots + \beta_KR^K(i,t)+ \varepsilon(i,t),
  \]
  where $\beta_k$ terms capture program spillovers, and $\gamma_k$ terms are
  simply fixed effects.
}


From the above, we have partitioned $X_i$ into $K$ different groups. However, we
are still unable to say anything about the distance $d$ above which spillovers no 
longer occur. From assumptions \ref{Sass:PTC} and \ref{Sass:SUTVAs}, we do 
however know that $d<Kh$, implying that there are at least some units for whom 
spillovers do not occur.  From (\ref{Seqn:Runpack}) and the preceding logic, this 
suggests that $d$ can be recovered following the iterative procedure laid out 
below, so long as $R(i,t)=f(X(i,t))$ is montononic in $X$.

If we start by estimating a typical DD specification like (\ref{Seqn:cDDconc}),
our estimated treatment effect, which I now denote $\hat\alpha^0$ is:
\begin{equation}
\nonumber
\begin{split}
E[\hat\alpha^0]=\alpha + \beta_1\Bias{D(i,t)}{R^1(i,t)}
                                + \beta_2\Bias{D(i,t)}{R^2(i,t)}
                                + \\ \ldots
                                + \beta_K\Bias{D(i,t)}{R^K(i,t)}.
\end{split}
\end{equation}
If spillovers exist below some distance $d$, then $\Cov[D(i,t),R^k(i,t)]>0 \ \ 
\forall \ \ kh<d$, given that $D(i,t)$---the treatment status in a treated area%
---affects the close to treated status in nearby areas. If this is the case, and 
if spillovers work in the same direction as treatment, then 
$|E[\hat\alpha^0]|<|\alpha|$, implying that the estimated treatment 
effect will be attenuated by treatment spillover to the control group.  

We can then re-estimate (\ref{Seqn:cDDconc}), however now \emph{also} condition
out $R^1(i,t)$ prior to estimating $\alpha$.  Our resulting estimate, 
$\hat\alpha^1$, will have the expectation:
\begin{equation}
\nonumber
\begin{split}
E[\hat\alpha^1]=\alpha + \beta_2\Bias{D(i,t)}{R^2(i,t)}
                                + \ldots
                                + \beta_K\Bias{D(i,t)}{R^K(i,t)}.
\end{split}
\end{equation}
Once again, if spillovers exist and are of the same sign as treatment, then the
estimate $\hat\alpha^1$ will be attenuated, but not as badly as $\hat\alpha^0$ 
given that we now partially correct for spillovers up to a distance of $h$.  In 
this case: $|E[\hat\alpha^0]|<|E[\hat\alpha^1]|<|\alpha|$.  If, on the other 
hand, spillovers do not exist, then we will have that 
$|E[\hat\alpha^0]|=|E[\hat\alpha^1]|=|\alpha|$.  This leads to 
the following hypothesis test, where for efficiency reasons $\hat\alpha^0$
and $\hat\alpha^1$ are estimated by seemingly unrelated regression:
\[
H_0: \alpha^0=\alpha^1 \hspace{1cm}
H_1: \alpha^0\neq\alpha^1.
\]
From \citet{Zellner1962}, the test statistic has a $\chi^2_1$ distribution. If 
we reject $H_0$ in favour of the alternative, this indicates that partially 
correcting for spillovers affects the estimated coefficient $\alpha$, implying 
that spillovers occur at least up to distance $h$, and that further tests are 
required.  

Rejection of the null suggests that another iteration should be performed, this 
time removing $R^1(i,t)$ and $R^2(i,t)$ from the error term $\upsilon(i,t)$ in 
(\ref{Seqn:cDDconc}), and the corresponding parameter $\alpha^2$ be estimated.  
If spillovers do occur at least up to distance $2h$, we expect that 
$|E[\hat\alpha^0]|<|E[\hat\alpha^1]|<|E[\hat\alpha^2]|<|\alpha|$, however if 
spillovers only occur up to distance $h$, we will have 
$|E[\hat\alpha^0]|<|E[\hat\alpha^1]|=|E[\hat\alpha^2]|=|\alpha|$.  This leads 
to a new hypothesis test:
\[
H_0: \alpha^1=\alpha^2 \hspace{1cm}
H_1: \alpha^1\neq\alpha^2,
\]
where the test statistic is distributed as outlined above.  Here, rejection of 
the null implies that spillovers occur at least up to distance $2h$, while 
failure to reject the null suggests that spillovers only occur up to distance 
$h$.

%-------------------------------------------------------------------------------
This process should be followed iteratively up until the point that the marginal 
estimate $\hat\alpha^{k+1}$ is equal to the preceding estimate $\hat\alpha^{k}$.  
At this point, we can conclude that units at a distance of at least $kh$ from 
the nearest treatment unit are not affected by spillovers, and hence a 
consistent estimate of $\alpha$ can be produced. Finally, this leads to a 
conclusion regarding $d$ and the indicator function $C(i,t)=\mathbf{1}_{X(i,t)
\leq d}$.  When controlling for the marginal distance to treatment indicator no 
longer affects the estimate of the treatment effect $\alpha^k$, we can conclude 
that $d=kh$, and thus correctly identify $C(i,t)=\mathbf{1}_{X(i,t)\leq kh}$ in 
data.

%-------------------------------------------------------------------------------
%\subsection{Estimating the Magnitude of Spillovers}
%\label{Ssscn:SE}
Thus far in
n section \ref{Ssscn:TE}, I discuss the consistent estimation of $\alpha$, the
effect of being in a treatment area.  The extension of this methodology to 
consistently estimate $\beta$, the effect of being close to treatment, is 
reasonably straightforward.  Once the scalar value $d$ has been determined, and
with data $\{Y(i,t), D(i,t), X(i,t): i=1, \ldots, N, t=0, 1\}$ in hand, we can 
use $d$ to map $X(i,t)$ into $C(i,t)$. Given the above we can now estimate 
(\ref{Seqn:cDD}), and form consistent estimates $\hat\beta$ and $\hat\alpha$ 
using OLS.

The estimate $\hat\beta$ will be the average treated effect on the close to 
treated (ATC), and will be one summary value for all areas to which spillovers 
occur. However, more information regarding the precise manner of propogation can 
be observed by estimating with the re-parametrized $R(i,t)$ from 
(\ref{Seqn:Runpack}) instead of the indicator variable $C(i,t)$. This suggests 
an alternative spillover test, in the style of that proposed in section 
\ref{Ssscn:TE}.  Rather than observing $\hat\alpha^j$ at each stage of the 
estimation process, $\hat\beta^j$ can be directly observed. If 
$\hat\beta_j\neq 0$, this suggests that the effect on the marginal close to 
treatment area is different to the effect in the (remaining) control area. If 
spillovers are the estimand of interest, additional $R^j(i,t)$ controls can be 
added until the hypothesis: $H_0: \beta_j = 0$ cannot be rejected for the 
marginal parameter. The empirical illustrations in section \ref{Sscn:empirics} 
estimate both the treatment effect, as well as spillovers at varying distances 
from treatment.



%-------------------------------------------------------------------------------
\subsection{Determining Optimal Distance Bins}
Up to this point it is assumed that $h$, the distance partition parameter, is
either known, or in some other way exogenously given.  However, the choice
of $h$ involves a well-known trade-off between precision and bias.  In the
case that a very large value of $h$ is chosen, very local spillovers may be
concealed, and hence parameter estimates will be biased, while very small
values of $h$ will lead to imprecise estimates of spillover effects.

In order to optimally and non-arbitrarily determine the value of $h$ used in
spillover search, a data snooping procedure is suggested which minimises the
Root Mean Squared Error (RMSE) of estimation.  This minimum RMSE technique is
quite closely related to bandwidth search in regression discontinuity models
(see for example discussion in \citet{LudwigMiller2007,ImbensLemieux2008}). In
order to do this, we consider the following Cross-Validation (CV) function:
\[
CV(h)=\frac{1}{N}\sum_{i=1}^N (Y_i-\widehat{Y}^*(X_i;h,\widehat\theta_{-i}))^2.
\]
This CV function calculates, for all $i\in 1,\ldots,N$, the predicted value
$\widehat{Y}$ based on $i$'s realizations of $X=X_i$, a particular value of $h$,
and regression parameters estimated using all observations with the exception
of $i$ ($\widehat\theta_{-i}$).  This ``leave-one-out'' procedure\footnote{
  Calculating $CV(h)$ following the leave-one-out procedure may be
  computationally demanding when the number of observations $N$ is large.  We
  thus examine alternative cross-validation procedures, including $k$-fold
  cross-validation, and stratified $k$-fold cross-validation, which produce
  estimates for $CV(h)$ in a less computationally-demanding way.  We return
  to these considerations in section \ref{sscn:montecarlo}.} provides a
measure of the prediction error for each observation given a particular
bandwidth $h$.  In order to optimally choose $h$, we seek to minimize the CV
function:
\[
h_{CV}^* =\argmin_h CV(h),
\]
where here $h_{CV}^*$ is the optimal bandwidth as calculated from the
leave-one-out CV procedure.  We return to consider the properties of this
procedure under simulation in section \ref{sscn:montecarlo}.  It is important
to note that in the above CV procedure, the quantity $\widehat{Y}$ depends on
the degree that spillovers are captured using a particular bandwidth $h$.
Thus, prior to calculating $\widehat{Y}$, the spillover-robust procedure
described in section \ref{Ssscn:TE} is followed, with the given value of
$h$.

Interestingly, in the above procedure there is nothing which limits the
value of $h$ to be constant between iterations.  One could envisage a
situation in which all possible combinantions of $h$ were chosen at each
subsequent iteration, and hence rather than searching for a scalar $h^*$,
the data snooping procedure would search for a vector $\mathbf{h}^*_{CV}$.
However, computational complexity in this case increases when searching for
the entire series of $\mathbf{h}^*_{CV}$.  In particular, the algorithm
complexity increases from $O(N)$ to $O(N^2)$, where $N$ refers to the number
of observations for which the leave-one-out CV procedure must be performed.

%We follow \citet{PorterYu2014} in treating the cutoff point $d$ as a
%nuisance parameter which must be estimated.

%-------------------------------------------------------------------------------
\subsection{Estimating with Multidimensional Spillovers}
\label{Ssscn:multi}
Previously it has been assumed that $R(i,t)$ is a function of a unidimensional 
distance measure $X(i,t)$. I now generalise this to a multidimensional case 
where $R(i,t)$ may depend upon an arbitrary number of variables 
$\mathbf{X}(i,t)$. This allows for cases where distance to treatment may 
interact with some other variable, such as income, ownership of a vehicle or
access to information (among other things). Now:
\begin{equation}
\nonumber
 R(i,t) =
  \begin{cases}
   f\Big(\mathbf{X}(i,t)\Big)   & \text{if an individual resides close to, but not in, a treatment area} \\
   0                            & \text{otherwise} 
  \end{cases}
\end{equation}

In order to allow for spillovers to depend upon a range of observable variables,
we must generalise assumption \asref{Sass:SUTVAl}{A}.  In order to do this, the
following new terminology is introduced, following \citet{Zajonc2012}. An 
assignment rule, $\delta$, maps units with covariates $\mathbf{X=x}$ to close
assignment $r$:
\[
\delta: \mathcal{X} \rightarrow \{0,1\}.
\]
This leads to a close-to-treatment assignment set $\mathbb{T}$ defined as:
\[
\mathbb{T}\equiv \{ \mathbf{x}\in\mathcal{X}: \delta(\mathbf{x})=1 \}
\]
whose complement $\mathbb{T}^c$ is known as the control assignment
set. Finally then, we can write the treatment assignment rule\footnote{The
uni-dimensional case discussed up to this point is just a particular application
of the treatment assignment rule where $\mathbf{X}(i,t)=X(i,t)$ and 
$\mathbb{T}\equiv \{ x<d: \delta(x)=1 \}$}:
\begin{equation}
\delta(x)\equiv \mathbf{1}_{\mathbf{x}\in\mathbb{T}}.
\end{equation}
With this (multidimensional) treatment assignment rule in hand, a more general 
version of assumption \asref{Sass:SUTVAl}{A} can now be provided:

%\addtocounter{assumption}{-1}
\begin{assumption}{4}{B}
\label{Sass:SUTVAlM}
\textbf{Assignment to close to treatment depends on observable $\mathbf{X}(i,t)$:} \\ 
An multidimensional assignment rule $\delta(x)=\mathbf{1}_{\mathbf{x}\in \mathbb{T}}$ 
exists which maps individuals to close to treatment status $C(i,t)$, where 
$\mathbf{X(i,t)}$ are observed covariates, and $\mathbb{T}$ is a fixed 
%but unknown (though reveal this and how to estimate it later)
function of $\mathbf{X(i,t)}$.
\end{assumption}

\begin{proposition}
\label{Pass:ATTnonP}
%-------------------------------------------------------------------------------
Under assumptions \ref{Sass:PT}--\ref{Sass:SUTVAs} and \asref{Sass:SUTVAlM}{B}, 
the ATT and ATC can be consistently estimated by least squares when controlling, 
parametrically or non-parametrically, for $C(i,t)=\mathbf{1}_{\mathbf{x}\in
\mathbb{T}}$. 
\end{proposition}
\noindent Refer to appendix \ref{app:proof} for proof. $\qed$

Now, in the same manner, we can go about generating our estimands of interest, 
replacing $C(i,t)=\mathbf{1}_{X_i\leq d}$ with $C(i,t)=\mathbf{1}_{\mathbf{x}\in 
\mathbb{T}}$. The most compuationally demanding step in this estimation procedure 
is in forming a parametric or non-parametric version of the underlying function 
$R(i,t)$ over which to search.  In a unidimensional framework it is reasonably 
straightforward to form local linear bins for $R(i,t)$.  However, in the 
multidimensional framework this is no longer the case.  Additionally, as the 
dimensionality of $\mathbf{X}$ rises, the number of search dimensions for 
spillovers also rises, leading to curse of dimensionality type considerations in 
the estimation of $\alpha$.

The particular functional form assigned to $R(i,t)$ will be context-specific,
and ideally driven by economic theory.  As mode of example, below we consider the
case where $R(i,t)=f(X_1,X_2)$ is a function of two variables, one binary and
the other continuous.  Such a case would be appropriate for a situation in which
spillovers depend upon distance to treatment and some indicator, such as exceeding
some income threshold.  Consider the case where $X_1\in \{0,1\}$ is binary, and 
$X_2$ continuous.  Then we can parametrise $R(i,t)$ as:
\begin{eqnarray}
R(i,t)&=&f(X_1,X_2) \nonumber \\
      &=&X_1\cdot[\beta_{0,1}X_2^1(i,t)+ \cdots + \beta_{0,K}X_2^K(i,t)] \nonumber \\
      &+& (1-X_1)\cdot[\beta_{1,1}X_2^1(i,t)+ \cdots + \beta_{1,K}X_2^K(i,t)]. \nonumber
\end{eqnarray}
where $X_2^k(i,t) \forall k \in 1\ldots K$ is defined as per (\ref{Seqn:Rpar}).  
Estimation of $\alpha$ 
can then proceed iteratively as in section \ref{Ssscn:TE}.  First a traditional 
DD parameter is estimated ignoring the possiblity that spillovers exist, leading 
to the proposed estimate $\hat\alpha^0$.  Then $X_1\cdot[\beta_{0,1}X_2^1(i,t)]$ 
and $(1-X_1)\cdot[\beta_{1,1}X^1(i,t)]$ are included in the regression, leading to 
an updated estimate $\hat\alpha^1$.  If the hypothesis $H_0: \alpha^0=\alpha^1$ 
cannot be rejected this suggests that spillovers are not a relevant phenomenon 
for either group, and the estimate of $\hat\alpha^0$ is accepted as the ATT.  
Otherwise, an additional iteration is made until the inclusion of the marginal 
$X_2^k(i,t)$ indicators for $X_1 \in \{0,1\}$ no longer affect the estimated 
effect $\alpha^k$.


%\section{Extensions}
%\label{Sscn:extend}

%\subsection{Semi-Parametric Estimation of Treatment Effects}
%Robinson style back fit to get the coefficient on treatment (but not close).
%\citet{Imbens2004}.  Just Monte Carlo this.

\section{Results}
To examine the performance of the spillover-robust DD estimator proposed
in section \ref{Sscn:estim}, we first examine recovered estimates under
a range of (known) data generating processes (DGPs) via Monte Carlo
simulation in section \ref{sscn:montecarlo}. We then turn to an extended
empirical example in section \ref{sscn:roadsTexting}, where we examine
whether state-level text messaging bans have local spillovers over roadways,
given that driving behaviour does not update immediately at state borders.

\subsection{Monte Carlo Evidence}
\label{sscn:montecarlo}
We conduct a series of estimates under simulation, where we are principally
interested in two considerations: the first, does the proposed estimation
strategy adequately capture the nature of spillovers, and secondly, does
this procedure allow us to correctly recover good estimates of the impact of
treatment when naive control groups would otherwise be locally contaminated.
In order to do so, we focus on a number of alternative DGPs.  These are
chosen as they will allow us to examine the estimand's performance both when
the spillover bins are correctly specified, and when they are unable to
precisely capture the geographical nature of spillovers (ie, in cases of
model mis-specification).

We consider first a model which is entirely amenable to estimation following
the proposed spillover robust DD methodology.  This first model is:
\[
y_{it} = \alpha + \beta T_{it} + \sum_{j=1}^4\gamma_jclose_{it,((j-1)\times5,j\times5]} + \phi_t + \lambda_i + \varepsilon_{it},
\]
where the outcome of interest is a function of treatment receipt ($T_{it}$),
as well as four spillover indicators, which capture spillovers occurring in
bins of 5 units of distance, and are indicated in each variable's subscript.
These are defined to be mutually exclusive, and
to refer to units between (0,5] units from treatment, from (5,10], from (10,15]
and from (15,20].  Treatment is a binary indicator, fixed to be equal to 1
for 20\% of the sample, and distance is simulated to allow for the examination
of estimated parameters when spillovers occur to 5, 10 or 25\% of the population.
The difference-in-difference structure of the model is captured with the inclusion
of a time fixed effect ($\phi_t$) and unit fixed effects ($\gamma_i$), where
we consider two time periods, with treatment switched on for treated units in
period 2.  We define treatment effects $\beta$ and close to treatment effects
$\bm{\gamma}$ to decrease as the distance from treatment increases.  Specifically
$\bm{\theta}=(\beta,\gamma_1,\gamma_2,\gamma_3,\gamma_4)=(10,5,4,3,2)$,
and, $\varepsilon\sim\mathcal{N}(0,\sigma)$, where $\sigma$ is allowed to vary
between simulations, taking values of 1, 2 or 5.

Model 1, described above, is particularly amenable for estimation using
the spillover-robust DD model given that spillovers are linear in nature,
and are demarcated in constant bins of 5 geographic units.  We thus consider
two alternative models to examine the performance of the esimator where
the spillover indicators will be, by necessity, mis-specified.  The first
consider spillovers which are linear, however occur in irregular distance
intervals.  Specifically, the outcome is generated as: 
\[
y_{it} = \alpha + \beta T_{it} + \sum_{j\in(0,2],(2,9],(9,16],(17,20]}\gamma_jclose_{itj} + \phi_t + \lambda_i + \varepsilon_{it}
\]
where now close to treatment effects are assigned in distances of
(0,2],(2,9],(9,16] and (17,20] from treatment.  Finally, we consider
a specification where treatment 

Model 3:
\[
y_{it}=\begin{cases}
\alpha + \beta T_{it} + \gamma \exp{(-dist)}+ \phi_t + \lambda_i+\varepsilon_{it} \qquad \text{if\ } 0 < dist \leq 10   \\
\alpha + \beta T_{it} + \varepsilon_{it}+ \phi_t + \lambda_i \qquad \text{otherwise}.
\end{cases}
\]

Cross validation search for $h^{*}$: Appears to have good size under Monte
Carlo simulation for ATE where the spillover distance is treated as a nuisance
parameter.  However, Given that there is additional sampling uncertainty
with respect to $h$, if we are interested in estimating the ATC, test statistics
do not have good size properties in analytic standard errors.  Suggestion:
bootstrap standard errors.


\subsection{An Applied Example: Test Messaging Bans and Local Spillovers over
  Roadways}
\label{sscn:roadsTexting}
\citet{AboukAdams2013}

\begin{itemize}
\item Show esitmates at state and municipal level
\item Show un-specified spillovers
\item Show algorithm generate spillovers (with RMSE, h)
\item Graph of accidents by distance?
\end{itemize}

%================================================================================ 
\section{Conclusion}
Echoing \citet{Bertrandetal2004}, ``Differences-in-Differences (DD) estimation 
has become an increasingly popular way to estimate causal relationships''.  
It is important to consider the assumptions underlying these estimators.  
In this paper we examine how DD estimates perform when the stable unit treatment 
value assumption does not hold locally.  Such a situation may be common in 
estimates of the causal effect of policy where compliance is imperfect. If 
policies entail a benefit to recipients, and if recipients living `close to' 
treatment areas who are themselves untreated can somehow cross regional 
boundaries to receive treatment, we may be concerned that, locally at least, 
SUTVA is violated.

In this paper I derive a set of conditions by which DD estimates can produce 
unbiased estimates even in the absence of the SUTVA holding between all units.  
It is shown that under a weaker set of conditions, both the average effect on the 
treated and the average effect on the `close to treated' can be estimated in a 
DD-type framework.  It is suggested that in the absence of this correction for 
local violations of SUTVA that (if spillovers actually \emph{do} occur) the true 
effect of the policy is likely to be attenuated.  

Using two empirical examples from recent contraceptive policy expansions, it is 
shown that this is---at least in these cases---an important consideration for 
the estimation of treatment effects, and effects on nearby neighbourhoods.  For
both Chile and Mexico, it is shown that pregnancy rates in neighbourhoods 
located close to areas where contraceptive reforms took place had subsequent
reductions in rates of teenage pregnancy.  What's more, in Chile (but not in 
Mexico), the correction for spillovers results in a significant reduction in
estimated treatment effects on the treated, correcting an attenuation bias when
control units are partially treated.  This is a useful reflection on this 
methodology: where treatment is geographically disperse, and hence many people
live close to treatment areas (as in Chile), correcting for failures of the
SUTVA is likely to be particularly important.  In cases where treatment is only
available in a reduced geographic area (such as Mexico), the degree of
importance of spillovers are likely to be considerably less when considering
estimates of average effects on treated areas.

These tests are easy to run, and indeed a software package that automates this
methodology is released with this paper.  Given the nature of the assumptions
underlying identification in many DD models in the literature, tests of this 
nature should be included in a basic suite of falsification tests.  While the
examples in this paper are illustrated using geographic spillovers, spillover-%
robust DD estimation is certainly not limited to only geographic cases.  How
(and whether) treatment travels between units should be of fundamental concern 
to many applications in the economic literature.
\newpage

\bibliography{ThesisRefs}
\newpage

\begin{landscape}
\section*{Figures and Tables}
\begin{figure}[htpb!]
  \centering
  \begin{subfigure}[b]{0.325\textwidth}
    \centering
    \includegraphics[width=\textwidth]{figures/RMSEplot_r1_sim2.eps}
    \caption{\small DGP 1}\label{fig:DGP1}
  \end{subfigure}
  \begin{subfigure}[b]{0.325\textwidth}
    \centering
    \includegraphics[width=\textwidth]{figures/RMSEplot_r4_sim2.eps}
    \caption{\small DGP 2}\label{fig:DGP2}
  \end{subfigure}
  \begin{subfigure}[b]{0.325\textwidth}
    \centering
    \includegraphics[width=\textwidth]{figures/RMSEplot_r7_sim2.eps}
    \caption{\small DGP 3}\label{fig:DGP3}
  \end{subfigure}
  \caption{Root Mean Squared Error and Bandwidth Search}
          \label{fig:rootMSE}
          \floatfoot{\textsc{Notes to figure \ref{fig:rootMSE}}: Root
            mean squared error (RMSE) under various data generating processes
            is displayed, allowing for spillovers calculated using
            bandwidth $h$.  Gray solid lines present alternative simulations
            (250 simulations shown here), and the solid green line with
            circles documents the average RMSE for each bandwidth over the
            full set of simulations.  DGPs are laid out fully in section
            \ref{sscn:montecarlo}.  For each simulation, $N=1,000$ observations
            and $\varepsilon\sim\mathcal{N(0,1)}$.  20\% of the sample are treated
            units, and 10\% of the sample are ``close to treated'' in the original
            DGPs. Identical plots for alternative degrees of spillovers are
            displayed in Appendix Figure \ref{fig:rootMSE-A}. The RMSE calculated
            for each bandwidth shown is based on the iterative procedure described
            in the text, with the optimal spillover-robust DD model corresponding
            to that estimated using the bandwidth which returns the lowest RMSE.
            The minimum RMSE and the minimum bandwidth associated with this RMSE
            (ie the optimal model bandwidth) is displayed in each panel.}
\end{figure}
\end{landscape}

\input{../tables/MC-treat.tex}

\begin{figure}[htpb!]
  \centering
  \begin{subfigure}[b]{0.475\textwidth}
    \centering
    \includegraphics[width=\textwidth]{figures/minDistanceAll.eps}
    \caption[All Texting Bans]%
            {{\small All Texting Bans}}
            \label{fig:all}
  \end{subfigure}
  \hfill
  \begin{subfigure}[b]{0.475\textwidth}
    \centering
    \includegraphics[width=\textwidth]{figures/minDistanceHH.eps}
    \caption[]%
            {{\small Handheld Bans}}
            \label{fig:HH}
  \end{subfigure}
  \vskip\baselineskip
  \begin{subfigure}[b]{0.475\textwidth}
    \centering
    \includegraphics[width=\textwidth]{figures/minDistanceStrong.eps}
    \caption[]%
            {{\small Strong Bans}}
            \label{fig:strong}
  \end{subfigure}
  \quad
  \begin{subfigure}[b]{0.475\textwidth}
    \centering
    \includegraphics[width=\textwidth]{figures/minDistanceWeak.eps}
    \caption[]%
            {{\small Weak Bans}}
            \label{fig:weak}
  \end{subfigure}
  \caption[ Distances between Counties and Treatment States ]
          {\small Distances between Counties and Treatment States}
          \label{fig:distance}
          \floatfoot{\textsc{Notes to figure \ref{fig:distance}}: Each
            panel displays distances of each county to the nearest
            treatment state once all bans have been enacted. States
            indicated in red are those treated in each case.  Panel (a)
            displays distances to any types of bans, panel (b) displays
            cases with a universal concurrent hand-held ban, panel (c)
            displays only bans with primary enforcement, and panel (d)
            dispalys bans only with secondary enforcement. Distances are
            displayed by county, based on the distance from the centre
            of each county to the closest point on the border of the
            closest treated state.}
\end{figure}
\clearpage

\begin{table}
  \begin{center}
    \caption{Reform and Spillover Effects: Bins in 10km}
    \scalebox{0.9}{
    \begin{tabular}{lcccccc}
      \toprule
      &(1)&(2)&(3)&(4)&(5)\\ \midrule
      \multicolumn{6}{l}{\textbf{Panel A: Strong Ban}} \\
      \input{../tables/Spillover_manual_strong.tex} 
      \multicolumn{6}{l}{\textbf{Panel B: Weak Ban}} \\
      \input{../tables/Spillover_manual_weak.tex} 
      \multicolumn{6}{l}{\textbf{Panel C: Handheld Ban}} \\
      \input{../tables/Spillover_manual_handheld.tex}
      \bottomrule
      \multicolumn{6}{p{15cm}}{{\footnotesize \textsc{Notes}: Each panel
          presents a separate difference-in-differences model with
          progressive controls capturing time-varying distance to treatment.
          Distances are arbitrarily split into bands of 10km.  Controls
          follow the specifications described in \citet{AboukAdams2013}
          exactly. Standard errors are clustered by state, and observations
          are weighted by county population. Each specification includes
          county by state fixed effects and month by year fixed effects.
          The dependent variable is the natural logarithm of fatal accidents
          + 1.}}
    \end{tabular}}
  \end{center}
\end{table}
\thispagestyle{empty}


\begin{table}
  \begin{center}
    \caption{Spillover Robust Difference-in-Difference Estimates}
    \begin{tabular}{lccc}
      \toprule
      &(1)&(2)&(3)\\ 
      &Strong & Weak & Handheld \\
      &Ban&Ban&Ban \\ \midrule
      \input{../tables/SpilloverDifDif.tex}
      \bottomrule
      \multicolumn{4}{p{13.3cm}}{{\footnotesize \textsc{Notes}: Each
          column presents a single spillover-robust
          difference-in-difference model.  Optimal models are
          based on minimising the RMSE criterion, with the optimal
          cross-validated RMSE displayed at the foot of the table.
          Spillover bins ($h$) are searched ranging from 2km to 40km,
          based on average distances from counties to the nearest
          treated state border. Optimal bandwidth, and maximum spillover
          distances in optimal models are displayed in the table footer.}}
    \end{tabular}
  \end{center}
\end{table}
\clearpage

%-------------------------------------------------------------------------------
\setcounter{table}{0}
\renewcommand{\thetable}{A\arabic{table}}
\setcounter{figure}{0}
\renewcommand{\thefigure}{A\arabic{figure}}


\appendix
\section*{Appendix Figures and Tables}
\begin{figure}[htpb!]
  \centering
  \begin{subfigure}[b]{0.325\textwidth}
    \centering
    \includegraphics[width=\textwidth]{figures/RMSEplot_r1_sim1.eps}
    \caption{\small DGP 1 (5\% Spillovers)}\label{fig:DGP1a}
  \end{subfigure}
  \begin{subfigure}[b]{0.325\textwidth}
    \centering
    \includegraphics[width=\textwidth]{figures/RMSEplot_r1_sim2.eps}
    \caption{\small DGP 1 (10\% Spillovers)}\label{fig:DGP1b}
  \end{subfigure}
  \begin{subfigure}[b]{0.325\textwidth}
    \centering
    \includegraphics[width=\textwidth]{figures/RMSEplot_r1_sim3.eps}
    \caption{\small DGP 1 (25\% Spillovers)}\label{fig:DGP1c}
  \end{subfigure}
  \vspace{8mm} \\
  \begin{subfigure}[b]{0.325\textwidth}
    \centering
    \includegraphics[width=\textwidth]{figures/RMSEplot_r4_sim1.eps}
    \caption{\small DGP 2 (5\% Spillovers)}\label{fig:DGP2a}
  \end{subfigure}
  \begin{subfigure}[b]{0.325\textwidth}
    \centering
    \includegraphics[width=\textwidth]{figures/RMSEplot_r4_sim2.eps}
    \caption{\small DGP 2 (10\% Spillovers)}\label{fig:DGP2b}
  \end{subfigure}
  \begin{subfigure}[b]{0.325\textwidth}
    \centering
    \includegraphics[width=\textwidth]{figures/RMSEplot_r4_sim3.eps}
    \caption{\small DGP 2 (25\% Spillovers)}\label{fig:DGP2c}
  \end{subfigure}
  \vspace{8mm} \\
  \begin{subfigure}[b]{0.325\textwidth}
    \centering
    \includegraphics[width=\textwidth]{figures/RMSEplot_r7_sim1.eps}
    \caption{\small DGP 3 (5\% Spillovers)}\label{fig:DGP3a}
  \end{subfigure}
  \begin{subfigure}[b]{0.325\textwidth}
    \centering
    \includegraphics[width=\textwidth]{figures/RMSEplot_r7_sim2.eps}
    \caption{\small DGP 3 (10\% Spillovers)}\label{fig:DGP3b}
  \end{subfigure}
  \begin{subfigure}[b]{0.325\textwidth}
    \centering
    \includegraphics[width=\textwidth]{figures/RMSEplot_r7_sim3.eps}
    \caption{\small DGP 3 (25\% Spillovers)}\label{fig:DGP3c}
  \end{subfigure}
  \caption{Root Mean Squared Error and Bandwidth Search Varying Degree of Spillovers}
          \label{fig:rootMSE-A}
          \floatfoot{\textsc{Notes to figure \ref{fig:rootMSE-A}}: Refer to Figure
            \ref{fig:rootMSE} for full notes.  Here we provide full RMSE plots
            for a range of degree of spillovers.  The three figures in the left-hand
            column are based on 5\% of the sample being affected by spillovers (for
            each of the three DGPs examined), the middle column are based on 10\%
            spillovers (reproduced for comparison from Figure \ref{fig:rootMSE}),
            and the right-hand panel are based on 25\% of the sample being impacted
            by spillovers.}
\end{figure}

\begin{landscape}
\begin{figure}[htpb!]
  \centering
  \begin{subfigure}[b]{0.325\textwidth}
    \centering
    \includegraphics[width=\textwidth]{figures/RMSEmean_r1_sim2.eps}
    \caption{\small DGP 1}\label{fig:RMSECVkfold1}
  \end{subfigure}
  \begin{subfigure}[b]{0.325\textwidth}
    \centering
    \includegraphics[width=\textwidth]{figures/RMSEmean_r4_sim2.eps}
    \caption{\small DGP 2}\label{fig:RMSECVkfold2}
  \end{subfigure}
  \begin{subfigure}[b]{0.325\textwidth}
    \centering
    \includegraphics[width=\textwidth]{figures/RMSEmean_r7_sim2.eps}
    \caption{\small DGP 3}\label{fig:RMSECVkfold3}
  \end{subfigure}
  \caption{Root Mean Squared Error Criterion with $k$-fold versus Leave-One-Out Cross-Validation}
          \label{fig:RMSECVkfold}
          \floatfoot{\textsc{Notes to figure \ref{fig:RMSECVkfold}}: Root
            Mean Squared Error (RMSE) by spillover bandwidth is plotted for
            Leave-One-Out (LOO) cross-validation (line with circles) and $k$-fold
            cross-validation (line with squares).  The $k$-fold cross validation
            is calculated using 10 folds, where folds are stratified by
            spillover bins, to allow approximate balance of spillover variables
            in each of the $k$ folds.  The optimal bandwidth distance ($h^{*}$)
            for each of the LOOCV and k-fold CV procedure are displayed in each
            panel of the figure, and DGPs are exactly as shown in identical panels
            of Figure \ref{fig:rootMSE}.}
\end{figure}
\end{landscape}

\input{../tables/MC-treat-kfold.tex}

\begin{table}[htpb!]
  \begin{center}
    \caption{Summary Statistics: State and Municipal-Level}
    \begin{tabular}{lccccc} \toprule
      &&\multicolumn{3}{c}{Treatment Areas} \\ \cmidrule(r){3-5}
      &Control Areas & All months & Pre-ban & Post-ban\\ \midrule
      \multicolumn{5}{l}{\textbf{Panel A: State-level Data}} \\
      \input{../tables/StateSums.tex}
      \multicolumn{5}{l}{\textbf{Panel B: County-level Data}} \\
      \input{../tables/CountySums.tex} \bottomrule
      \multicolumn{5}{p{16cm}}{{\footnotesize \textsc{Notes}:
          Treatment and control states follow classifications provided in
          \citet{AboukAdams2013}. Panel A presents original state-level
          data used in difference-in-differences estimates presented in
          \citet{AboukAdams2013}.  Panel B presents county-level data
          used in spillover robust diff-in-diff estimates examined in this
          paper.  All variables follow precisely the definitions from
          original analysis.}} 
    \end{tabular}
  \end{center}
\end{table}

\input{../tables/MunicipalReplicate.tex}

\begin{table}[htpb!]
  \begin{center}
    \caption{State and Municipal-Level Baseline Difference-in-Difference Model}
    \begin{tabular}{lccc}
      \toprule
      &(1)&(2)&(3) \\
      \midrule
      \multicolumn{4}{l}{\textbf{Panel A: State-level Estimates}} \\
      \input{../tables/StateDifDif.tex} \\
      \multicolumn{4}{l}{\textbf{Panel B: County-level Estimates}} \\
      \input{../tables/CountyDifDif.tex}
      \bottomrule
      \multicolumn{4}{p{7.8cm}}{{\footnotesize \textsc{Notes}:
          Baseline difference-in-difference models without spillover estimates
          are presented.  Panel A presents state-level models following
          \citet{AboukAdams2013}. Panel B presents identical specifications
          however at the county level, and weighted by county, rather than
          by states.  Both models cluster standard errors by state, to allow
          arbitrary correlations among counties within states across time and
          across space.}} 
    \end{tabular}
  \end{center}
\end{table}

\begin{landscape}
\begin{figure}
  \begin{center}
    \caption{Geographical Distribution of Single Vehicle Single Occupant Accidents}
    \includegraphics[scale=1.25]{figures/geoDist_07-10.eps}
  \end{center}
\end{figure}
\end{landscape}

\clearpage


\section{Alternative Methodologies}
\label{app:literature}
The present paper is interested in quantifying the impacts of some externally
defined policy of interest, on individuals in treated areas, and on individuals
living in areas close to treatment.  The definition of the policy itself is
assumed to not be under the direct control of the treated and untreated
individuals themselves.  The setting of this paper is thus different to analyses
where externalities occur due to coordination of individual behaviour, and
spillovers from social interactions, such as conformation to reference group
behaviour (for example, as in \citet{BrockDurlauf2001,Benabou1993}.

\paragraph{Generalised Methods} The work of \citet{Manski2013,
  AronowSamii2017} provide the most general considerations of social
interactions in econometric models.  Both of these provide models allowing
for arbitrary forms of interdependence in treatment assignment between
individuals, and document the identification requirements on treatment
effects.  When considering the nature of spillovers, \citet{AronowSamii2017}
refer to an ``exposure mapping'', which describes individual treatments and
the propogation of treatment, while \citet{Manski2013} refers to ``effective
treatment'', to describe the same process. Importantly, neither model
requires that spillovers are limited to an individual's reference group,
however such a circumstance is nested in their models. In this paper, I also
do not require spillovers to be isolated to an individual's reference group,
meaning that the method proposed can be couched in both \citet{Manski2013} or
\citet{AronowSamii2017}'s terms.  The difference between this paper and
the work of \citet{Manski2013,AronowSamii2017} is that I focus on a particular
type of spillovers, fully specify the manner in which interference occurs,
and provide a systematic way to estimate the degree of interference.
This can thus be considered an estimable application of these theories to
capture geographic spillovers in a difference-in-difference setting.

In \citet{Manski2013}'s terms, the model laid out here allows for
monotone metric interactions, which can either be reinforcing (in the
same direction as receipt of treatment) or opposing (in the opposite
direction of receipt of treatment).  This is generally known as a
Semi-Monotone Treatment Response model in \citet{Manski2013}. The
spillover-robust diff-in-diff model laid out here lays out restrictions
to the shape of the ``response function'' (the degree to which treatment
propogates over space), without distributional assumption or direction
assumptions on this response function.  In \citet{AronowSamii2017}'s
terms, the spillover-robust DD model proposes a non-parametric ``exposure
mapping'' determining the treatment (and close to treatment) assignment
based on any original treatment assignment.  While \citet{AronowSamii2017}
propose an exposure mapping which is based on a generalised propensity
score for the likelihood of exposure with restrictions related to maximal
propogation based on sample size (their ``Local dependence'' condition),
the exposure mapping described in the spillover robust DD model proposes
an exposure mapping based on distance to treatment (and potential interactions).

%MANSKI (2013): Structural function defines outcome for all
% possible points on response function (refer to page S15 to check
% this)
%
% Response function: maps treatment vectors of all T in the population
% into outcomes.
%
% In general, my proposal seems to be a monotone metric interaction
% either reinforcing or opposing.  This is generally the Semi-Monotone
% Treatment Response models from Manski 2013.  The paper also restricts
% the shape of the response functions, without distributional assumption
% or direction assumptions on the response function.

\paragraph{Within but not between groups} \citet{HudgensHalloran2008,LiuHudgens2014}
Also known as hierarchical treatment assignment.

\paragraph{Similar Ideas, Alternative Methods} \citet{Rosenbaum2007}

\paragraph{Review} \citet{TchetgenTchetgenVanderWeele2012,
  AngelucciDiMaro2010,Blumeetal2011}

\paragraph{Applications} \citet{MiguelKremer2004}

\paragraph{Design to Estimate} \citet{Bairdetal2014}

\paragraph{Alternative Literature} \citet{ChaisemartinDHaultfoeuille2017}

\clearpage


\section{Proofs}
\label{app:proof}
%\renewcommand{\qedsymbol}{$\blacksquare$}
\begin{proof}[Proof of Proposition \ref{Pass:ATT}]
\begin{footnotesize}
$Y(i,t)$ is generated according to (\ref{Seqn:COV}), and from (\ref{Seqn:cDD}),
a regression of $Y(i,t)$ on $D(i,t)$ and $C(i,t)$ can be estimated.  It is 
assumed that we have at a representative sample of size $N$ consisting of 
$\{Y(i,t),D(i,t),X(i,t): i=1, \ldots, N, t=0, 1\}$.  By assumption 
\asref{Sass:SUTVAl}{A}, the assignment rule $\delta$ forms $C(i,t)$ allowing for 
the estimation of (\ref{Seqn:cDD}). By definition, $\alpha$ in this regression 
is equal to:
\begin{equation}
\nonumber
\label{Seqn:alphaProof1}
\begin{split}
\alpha&=\{E[Y(i,1)|D(i,1)=1,R(i,1)=0]-E[Y(i,1)|D(i,1)=0,R(i,1)=0]\} \\
      &-\{E[Y(i,0)|D(i,1)=1,R(i,1)=0]-E[Y(i,0)|D(i,1)=0,R(i,1)=0]\},
\end{split}
\end{equation}
and from assumption \ref{Sass:SUTVAs}, each of the expectation terms exists, as 
there are both fully treated and completely untreated units. Using the potential 
outcomes framework, we are free to re-write the above expression as:
\begin{equation}
\nonumber
\label{Seqn:alphaProof2}
\begin{split}
\alpha&=\{E[Y^1(i,1)|D(i,1)=1,R(i,1)=0]-E[Y^0(i,1)|D(i,1)=0,R(i,1)=0]\} \\
      &-\{E[Y^0(i,0)|D(i,1)=1,R(i,1)=0]-E[Y^0(i,0)|D(i,1)=0,R(i,1)=0]\},
\end{split}
\end{equation}
given that only in the case where $t=1$ and $D(i,1)=1$ we observe the potential 
outcome where the individual receives treatment: $Y^1(i)$.  Using the linearity
of the expectations operator, this can finally be re-written as:
\begin{equation}
\nonumber
\label{Seqn:alphaProof3}
\alpha=E[Y^1(i,1)-Y^0(i,0)|D(i,1)=1,R(i,1)=0] - 
       E[Y^0(i,1)-Y^0(i,0)|D(i,1)=0,R(i,1)=0].
\end{equation}

Now, from assumption \ref{Sass:PT}, we can appeal to parallel trends, and 
replace the second expectation term in the above expression with $E
[Y^0(i,1)-Y^0(i,0)|D(i,1)=1,R(i,1)=0]$:
\begin{equation}
\nonumber
\label{Seqn:alphaProof4}
\alpha=E[Y^1(i,1)-Y^0(i,0)|D(i,1)=1,R(i,1)=0] - E[Y^0(i,1)-Y^0(i,0)|D(i,1)=1,R(i,1)=0].
\end{equation}
Expanding the expectations operator and cancelling out the second term in each of
the above items gives:
\begin{equation}
\nonumber
\label{Seqn:alphaProof5}
\alpha=E[Y^1(i,1)|D(i,1)=1,R(i,1)=0] - E[Y^0(i,1)|D(i,1)=1,R(i,1)=0].
\end{equation}
%-------------------------------------------------------------------------------
which finally, once again by the linearity of expectations, can be combined to 
give $\alpha=E[Y^1(i,1)-Y^0(i,1)|D(i,1)=1,R(i,1)=0]$, which can be 
rewritten as $\alpha=E[Y^1(i,1)-Y^0(i,1)|D(i,1)=1]$ given that 
$D(i,1)=1 \implies R(i,1)=0$.  Combining (\ref{Seqn:estimATT}) and $\alpha=
E[Y^1(i,1)-Y^0(i,1)|D(i,1)=1]$ we thus have that $\alpha=ATT$ as 
required.

Turning to the ATC, the same set of steps can be followed for $\beta$ on the 
coefficient $R(i,t)$, however now instead of assumption \ref{Sass:PT} we must
rely on parallel-trend assumption \ref{Sass:PTC}. This leads to $\beta=
E[Y^1(i,1)-Y^0(i,1)|R(i,1)\neq 1]$, and from (\ref{Seqn:estimATC}) 
and the previous expression it holds that that $\beta=ATC$.
\end{footnotesize}
\end{proof}

\begin{proof}[Proof of Proposition \ref{Pass:ATTnonP}]
\begin{footnotesize}
With the representative sample $\{Y(i,t), D(i,t), \mathbf{X}(i,t): i=1, \ldots, N, 
t=0, 1\}$, assumption \asref{Sass:SUTVAlM}{B} implies that $\mathbf{X}(i,t)$ can
be $C(i,t)$ using assignment rule $\delta$.  The remainder of the proof follows
the same steps as the proof for proposition \ref{Pass:ATT}.
\end{footnotesize}
\end{proof}

\clearpage
%-------------------------------------------------------------------------------

%\section{Measuring Distance to Treatment Clusters}
%\label{Sscn:distApp}
%Principal measures of distance from treatment is calculated by taking a 
%Euclidean distance from the centroid of non-treatment clusters, to the centroid 
%of the nearest cluster which did receive treatment. However, alternative 
%measures may more accurately capture the true distance of an individual to 
%treatment.  As a robustness check, two alternative measures of distance to 
%treatment are calculated and used.
%
%Firstly, I collated the shortest distance over roads from non-treatment to 
%treatment areas.  This was calculated using repeated calls to the Google 
%Distance Matrix API\footnote{Full details can be found at:
%\url{https://developers.google.com/maps/documentation/distancematrix/\#api\_key}.
%I have made the computational routine used available on the web at:
%\url{https://github.com/damiancclarke/spillovers/blob/master/source/distCalc/queryDist.py}.}, 
%which finds the shortest path over roads.  In the case of Chile, this requires 
%calculating the distances between all 346  municipalities ($346^2/2=59,858$ 
%distance pairs), while in the case of M\'exico this requires calculating only 
%the distance from each municipality outside of Mexico DF to each municipality 
%inside Mexico DF ($2457\times 16=39,312$).  Secondly, rather than distance in 
%kilometres, as in Euclidean or road distance, a measure of travel \emph{time} 
%was calculated.  As a proxy for total travel time, travel time by car was 
%caclulated between areas.  This was similarly generated using calls to Google 
%Maps, resulting in one value for each municipality pair.  In each case 
%``distance to treatment'' is then the minimum value to the nearest treatment 
%area, which varies by municipality and year.
%
%These alternative measures of distance do not majorly affect the quantitative 
%implication of findings in either Chile or Mexico.  Appendix figure 
%\ref{Sfig:ATTTime} is the analogue of figure \ref{Sfig:ChileAlpha},
%using travel time rather than Euclidean distance between municipalities as
%a measure of spillover distance.  Results from both figures suggest a 
%treatment effect of approximately -0.075 once accounting for spillovers of
%30 minutes travel time or 30km of distance respectively.  Regression results
%for all age groups and all measures did not result in significantly different
%estimates of the effect of treatment in any case.
%
%\clearpage

\section{Spillovers as a Nuisance Parameter and Treatment Effect Stability}
If we start by estimating a typical DD specification like (\ref{Seqn:cDDconc}),
our estimated treatment effect, which I now denote $\hat\alpha^0$ is:
\begin{equation}
\nonumber
\begin{split}
E[\hat\alpha^0]=\alpha + \beta_1\Bias{D(i,t)}{R^1(i,t)}
                                + \beta_2\Bias{D(i,t)}{R^2(i,t)}
                                + \\ \ldots
                                + \beta_K\Bias{D(i,t)}{R^K(i,t)}.
\end{split}
\end{equation}
If spillovers exist below some distance $d$, then $\Cov[D(i,t),R^k(i,t)]>0 \ \ 
\forall \ \ kh<d$, given that $D(i,t)$---the treatment status in a treated area%
---affects the close to treated status in nearby areas. If this is the case, and 
if spillovers work in the same direction as treatment, then 
$|E[\hat\alpha^0]|<|\alpha|$, implying that the estimated treatment 
effect will be attenuated by treatment spillover to the control group.  

We can then re-estimate (\ref{Seqn:cDDconc}), however now \emph{also} condition
out $R^1(i,t)$ prior to estimating $\alpha$.  Our resulting estimate, 
$\hat\alpha^1$, will have the expectation:
\begin{equation}
\nonumber
\begin{split}
E[\hat\alpha^1]=\alpha + \beta_2\Bias{D(i,t)}{R^2(i,t)}
                                + \ldots
                                + \beta_K\Bias{D(i,t)}{R^K(i,t)}.
\end{split}
\end{equation}
Once again, if spillovers exist and are of the same sign as treatment, then the
estimate $\hat\alpha^1$ will be attenuated, but not as badly as $\hat\alpha^0$ 
given that we now partially correct for spillovers up to a distance of $h$.  In 
this case: $|E[\hat\alpha^0]|<|E[\hat\alpha^1]|<|\alpha|$.  If, on the other 
hand, spillovers do not exist, then we will have that 
$|E[\hat\alpha^0]|=|E[\hat\alpha^1]|=|\alpha|$.  This leads to 
the following hypothesis test, where for efficiency reasons $\hat\alpha^0$
and $\hat\alpha^1$ are estimated by seemingly unrelated regression:
\[
H_0: \alpha^0=\alpha^1 \hspace{1cm}
H_1: \alpha^0\neq\alpha^1.
\]
From \citet{Zellner1962}, the test statistic has a $\chi^2_1$ distribution. If 
we reject $H_0$ in favour of the alternative, this indicates that partially 
correcting for spillovers affects the estimated coefficient $\alpha$, implying 
that spillovers occur at least up to distance $h$, and that further tests are 
required.  

Rejection of the null suggests that another iteration should be performed, this 
time removing $R^1(i,t)$ and $R^2(i,t)$ from the error term $\upsilon(i,t)$ in 
(\ref{Seqn:cDDconc}), and the corresponding parameter $\alpha^2$ be estimated.  
If spillovers do occur at least up to distance $2h$, we expect that 
$|E[\hat\alpha^0]|<|E[\hat\alpha^1]|<|E[\hat\alpha^2]|<|\alpha|$, however if 
spillovers only occur up to distance $h$, we will have 
$|E[\hat\alpha^0]|<|E[\hat\alpha^1]|=|E[\hat\alpha^2]|=|\alpha|$.  This leads 
to a new hypothesis test:
\[
H_0: \alpha^1=\alpha^2 \hspace{1cm}
H_1: \alpha^1\neq\alpha^2,
\]
where the test statistic is distributed as outlined above.  Here, rejection of 
the null implies that spillovers occur at least up to distance $2h$, while 
failure to reject the null suggests that spillovers only occur up to distance 
$h$.

%-------------------------------------------------------------------------------
This process should be followed iteratively up until the point that the marginal 
estimate $\hat\alpha^{k+1}$ is equal to the preceding estimate $\hat\alpha^{k}$.  
At this point, we can conclude that units at a distance of at least $kh$ from 
the nearest treatment unit are not affected by spillovers, and hence a 
consistent estimate of $\alpha$ can be produced. Finally, this leads to a 
conclusion regarding $d$ and the indicator function $C(i,t)=\mathbf{1}_{X(i,t)
\leq d}$.  When controlling for the marginal distance to treatment indicator no 
longer affects the estimate of the treatment effect $\alpha^k$, we can conclude 
that $d=kh$, and thus correctly identify $C(i,t)=\mathbf{1}_{X(i,t)\leq kh}$ in 
data.




%\section{A Structural Interpretation of Spillover-Robust Difference-in-Difference Estimation}
%The framework discussed in this paper can be motivated in terms of a simple
%structural model in the style of \citet{Roy1951}, or 
%\citet{HeckmanVytlacil2005}.\footnote{While the \citet{Roy1951} model is not
%cast in terms of causal effects, it does consider selection based on relative
%utility.  The so-called ``Generalised Roy Model'' laid out in 
%\citet{HeckmanVytlacil2005} is a more appropriate framework for this analysis.}
%To start, potential outcomes are written as:
%\[
%Y_0= \mu_0(X) + U_0
%\]
%and
%\[
%Y_1= \mu_1(X) + U_1.
%\]
%Treatment assignment $T=1$ and $T=0$ are observed.  Each individual decides
%whether to opt in to treatment $D$, given their realisation of $X$ and $T$,
%along with the costs of participation:
%\[
%C=\mu_c(Z,T)+U_c.
%\]
%Their decision rule depends upon net utility:
%\[
%D^{*}=Y_1-Y_0-C,
%\]
%and can be expressed as: $D=\mathbf{1}_{D^{*}\geq 0}$.  Structural estimation
%relies on the assumption $(X,Z,T) \independent (U_0,U_1,U_c)$.\footnote{Note
%that the parallel trend assumption is a linear form of this.}  Note that in 
%the above specification, the costs of participation are explicitly allowed 
%to depend upon treatment assignment status $T$. This is a cost associated with
%defying the treatment status.
%
%Attach functionl forms: $\mu_0(X)=X\beta_0$, $\mu_1(X)=X\beta_1$, 
%$\mu_c(Z,T)=Z\beta_{cz}+T\beta_{cT}$, and distributional assumptions for the
%unobserved components: $(U_0,U_1,U_c)\sim \mathcal{N}(0,\Sigma)$ 



\end{spacing}
\end{document}
