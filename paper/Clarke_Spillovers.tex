%********************************************************************************
\documentclass{article}


\usepackage{natbib}
\bibliographystyle{abbrvnat}
\bibpunct{(}{)}{;}{a}{,}{,}
\usepackage{setspace}
\usepackage{amsmath}
\usepackage{amsthm}
\newtheorem{assumption}{Assumption}

\setlength\parskip{0.25in}

\title{Estimating Difference-in-Differences in the Presence of Spillovers\footnote{
I thank participants in the Impact Evaluation Meeting at the Inter-American 
Development Bank for useful comments on this draft. Source code, including the
Stata module \texttt{cdifdif} is available at 
https://github.com/damiancclarke/spillovers.  Affiliation: Faculty of Economics, 
The University of Oxford, Manor Road, Oxford. Contact email: 
damian.clarke@economics.ox.ac.uk}}
\author{Damian Clarke}


\begin{document}

\maketitle

\begin{abstract}
I propose a method for difference-in-differences (DD) estimation in situations 
where the stable unit treatment value assumption is violated locally.  A flexible 
methodology is described to test for such spillovers, and to consistently estimate 
treatment effects in their presence.  This methodology is applied---both 
parametrically and semi-parametrically---to two empirical examples.
\end{abstract}

\newpage
\begin{spacing}{1.1}
%********************************************************************************
\section{Introduction}
Natural experiments often rely on territorial borders to estimate treatment 
effects.  These borders separate quasi-treatment from quasi-control groups with
individuals in one area having access to a program or treatment while those in 
another do not.  In cases such as these where geographic location is used to 
motivate identification, the stable unit treatment value assumption (SUTVA) is, 
either explicitly or implicitly, invoked.\footnote{The SUTVA has a long and 
interesting history, under various guises. \citet{Cox1958} refers to ``no 
interference between different units'', before \citet{Rubin1978} introduced the 
concept of SUTVA (the name SUTVA did not appear until \citet{Rubin1980}.  Recent 
work of \citet{Manski2013}, refers to this assumption as Individualistic 
Treatment Response (ITR).}

However, often territorial borders are porous.  Generally state, regional,
municipal, and village boundaries can be easily, if not costlessly, crossed.
Given this, researchers interested in using natural experiments in this way may
be concerned that the effects of a program in a treatment cluster may spillover 
into non-treatment clusters---at least locally.

Such a situation is in clear violation of the SUTVA's need that the treatment
status of any one unit must not affect the outcomes of any other unit.  In this 
paper we propose a methodology to deal with such spillover effects.  We
discuss how to test for local spillovers, and if such spillovers exist, how to 
estimate unbiased treatment effects.  It is shown that this estimation requires
a weaker condition than SUTVA: namely that SUTVA holds between \emph{some} units, 
as determined by their distance from the treatment cluster.  We show how to 
estimate treatment effects both parametrically and semi-parametrically, and then
propose a method to generalise the proposed estimator to a higher dimensional 
case where spillovers may depend, parametrically or non-parametrically, on a number 
of factors.

This methodology is then illustrated with 2 empirical examples.  We examine how
spillovers of reforms across municipal boundaries may contaminate `traditional' 
difference-in-differences (DD) estimators.  This is applied to two contraceptive
reforms, where individuals from contiguous or nearby areas can travel to a 
treatment region to access the reform.  It is shown that both the plausibly 
exogenous arrival of the morning after pill to certain municipalities of Chile, 
and abortion to certain districts in Mexico results in a reduction of births
in the given area, and in close-by quasi-control areas.  As a result, the
spillover-robust DD estimator propsed here flexibly captures this effect, 
correcting for any (local) spillover bias.



Ideas: \citet{McIntosh2008}, \citet{Bairdetal2014}, \citet{AngelucciDiMaro2010} 
Applications: \citet{AngelucciDeGiorgi2009}, \citet{Heckmanetal1998}, 
\citet{MiguelKremer2004}
Other: \citet{Imbens2004}, \citet{Heckmanetal1998b}, \citet{Manski2013}

%********************************************************************************
%SATURATE
\section{Methodology}
Define $Y(i,t)$ as the outcome for individual $i$ and time $t$.  The population
of interest is observed at 2 time periods, $t\in \{0,1\}$.  Assume that between
$t=0$ and $t=1$, some fraction of the population is exposed to a 
quasi-experimental treatment.  As per \citet{Abadie2005}, we will denote 
treatment for individual $i$ in time $t$ as $D(i,t)$, where $D(i,1)=1$ implies 
that the individual was treated, and $D(i,1)=0$ implies that the individual was
not directly treated.  Given that treatment only exists between periods 0 and 1,
$D(i,0)=0\ \forall\ i$.

It is shown by \citet{AshenfelterCard1985} that if the outcome is generated by
a component of variance process:
\begin{equation}
\label{Seqn:COV}
Y(i,t)=\delta(t) + \alpha D(i,t)+\eta(i)+\nu(i,t)
\end{equation}
where $\delta(t)$ refers to a time-specific component, $\alpha$ as the impact of 
treatment, $\eta(i)$ a component specific to each individual, and $\nu(i,t)$ as 
a time-varying individual (mean zero) shock, then a sufficient condition for 
identification (a complete derivation is provided by \citet{Abadie2005}) is:
\begin{equation}
\label{Seqn:ID}
P(D(i,1)=1|\nu(i,t))=P(D(i,1)=1) \ \forall\ t\in\{0,1\}.
\end{equation}
In other words, identification requires that selection into treatment does not
rely on the unobserved time-varying component $\nu(i,t)$.  If this condition 
holds, then the classical DD estimator provides an unbiased estimate of the
treatment effect:
\begin{equation}
\label{Seqn:DD}
\begin{split}
\alpha&=\{E[Y(i,1)|D(i,1)=1]-E[Y(i,1)|D(i,1)=0]\} \\
      &-\{E[Y(i,0)|D(i,1)=1]-E[Y(i,0)|D(i,1)=0]\}.
\end{split}
\end{equation}

Assume now, however, that treatment is not precisely geographically bounded.  
Specifically, imagine that those living in control areas `close to' treatment 
areas are able to access treatment, either partially or completely.  Such a 
case allows for a case where individuals `defy' their treatment status, by 
travelling or moving treated areas, or where spillovers from treatment areas 
is diffused through general equilibrium processes.  Define $R(i,t)$ where
\begin{equation}
 R(i,t) =
  \begin{cases}
   1   & \text{if an individual lives close to, but not in, a treatment area} \\
   0   & \text{otherwise} 
  \end{cases}
\end{equation}
As treatment occurs only in period 1, $R(i,0)=0$ for all $i$.  Similarly, as 
living in a treatment area itself excludes individuals from living `close to' 
the same treament area, $R(i,t)=0$ for all $i$ such that $D(i,t)=1$.

Generalising from (\ref{Seqn:COV}), now we assume that $Y(i,t)$ is generated 
by:
\begin{equation}
\label{Seqn:COV2}
Y(i,t)=\delta(t) + \alpha D(i,t)+\beta R(i,t)+\eta(i)+\nu(i,t)
\end{equation}
If we observe only $Y(i,t)$, $D(i,t)$ and $R(i,t)$, a sufficient condition for 
estimation now consists of (\ref{Seqn:ID}) and the following assumption: 
\begin{equation}
\label{Seqn:ID2}
P(R(i,1)=1|\nu(i,t))=P(R(i,1)=1) \ \forall\ t\in\{0,1\}.
\end{equation}
This requires that both treatment, and being close to treatment cannot depend 
upon individual-specific time-variant components. To see this, write 
(\ref{Seqn:COV2}), adding and subtracting $E[\eta(i)|D(i,1),
R(i,1)]$:
\begin{equation}
\label{Seqn:addsub}
Y(i,t)=\delta(t) + \alpha D(i,t)+\beta R(i,t)+E[\eta(i)|D(i,1),R(i,1)]+\varepsilon(i,t)
\end{equation}
where, following \citet{Abadie2005}, $\varepsilon(i,t)=\eta(i)-E[\eta(i)|D(i,1),R(i,1)]
+\nu(i,t)$.  We can write $\delta(t)=\delta(0)+[\delta(1)-\delta(0)]t$, and write
$E[\eta(i)|D(i,1),R(i,1)]$ as the weighted sum of the expectation of the 
individual-specific component $\eta(i)$ over treatment status and `close' status%
\footnote{$E[\eta(i)|D(i,1),R(i,1)]=E[\eta(i)|D(i,1)=0,R(i,1)=0]+(E[\eta(i)|D(i,1)=1,
R(i,1)=0]-E[\eta(i)|D(i,1)=0,R(i,1)=0])\cdot D(i,1)+(E[\eta(i)|D(i,1)=0,R(i,1)=1]-
E[\eta(i)|D(i,1)=0,R(i,1)=0])\cdot R(i,1)$.}.  Finally define $\mu$ as 
$\mu=E[\eta(i)|D(i,1)=0,R(i,1)=0]+\delta_0$, $\tau$ as $\tau=E[\eta(i)|D(i,1)=1,R(i,1)
=0]-E[\eta(i)|D(i,1)=0,R(i,1)=0]$, $\gamma$ as $\gamma=E[\eta(i)|D(i,1)=0,R(i,1)=1]-
E[\eta(i)|D(i,1)=0,R(i,1)=0]$ and $\delta$ as $\delta=\delta(1)-\delta(0)$.  Then 
from the above and (\ref{Seqn:addsub})
we have:
\begin{equation}
\label{Seqn:cDD}
Y(i,t)=\mu+\tau D(i,1) + \gamma R(i,1) + \delta t + \alpha D(i,t) + \beta R(i,t) + 
       \varepsilon(i,t).
\end{equation}
Notice that this equation now includes the typical DD fixed effects $\tau$ and $\delta$
and the double difference term $\alpha$.  However it also includes `close' analogues
$\gamma$ (an initial fixed effect), and $\beta$: the effect of being `close to' a 
treatment area.

From assumptions (\ref{Seqn:ID}) and (\ref{Seqn:ID2}) it holds that $E[(1,D(i,1),R(i,1),
D(i,t),R(i,t))\cdot\varepsilon(i,t)]=0$, which implies that all parameters from
(\ref{Seqn:cDD}) are consistently estimable by OLS.  Importantly, this includes
consistent estimates of $\alpha$ and $\beta$: the effect of the program treatment 
and spillover effects on outcome variable $Y(i,t)$.  Then, from (\ref{Seqn:cDD}),
a our coefficients of interest $\alpha$ and $\beta$ are:
\begin{equation}
\label{Seqn:DDa}
\begin{split}
\alpha&=\{E[Y(i,1)|D(i,1)=1,R(i,1)=0]-E[Y(i,1)|D(i,1)=0,R(i,1)=0]\} \\
      &-\{E[Y(i,0)|D(i,1)=1,R(i,1)=0]-E[Y(i,0)|D(i,1)=0,R(i,1)=0]\}, 
\end{split}
\end{equation}
and 
\begin{equation}
\label{Seqn:DDb}
\begin{split}
\beta&=\{E[Y(i,1)|D(i,1)=0,R(i,1)=1]-E[Y(i,1)|D(i,1)=0,R(i,1)=0]\} \\
      &-\{E[Y(i,0)|D(i,1)=0,R(i,1)=1]-E[Y(i,0)|D(i,1)=0,R(i,1)=0]\}. 
\end{split}
\end{equation}
where the sample estimate of each parameter is generated by a least squares
regression of (\ref{Seqn:cDD}) using a random sample of 
$\{Y(i,t), D(i,t), R(i,t): i=1, \ldots, N, t=0, 1\}$.

%********************************************************************************
\section{A Spillover-Robust Double Differences Estimator}
We are interested in estimating difference-in-difference parameters $\alpha$ and 
$\beta$ from (\ref{Seqn:cDD}).  We will refer to these estimators respectively
as the average treatment effect on the treated (ATT), and the average treatment
effect on the close to treated (ATC).  Average treatment effects are cast in 
terms of the \citet{Rubin1974} Causal Model.

Following a potential outcome framework, we denote $Y^1(i,t)$ as the potential
outcome for some person $i$ at time $t$ if they were to receive treatment, and
$Y^0(i,t)$ if the person were not to receive treatment.  Our ATT and ATC are
thus:
\begin{eqnarray}
\label{Seqn:estim}
ATT=E[Y^1(i,1)-Y^0(i,1)|D(i)=1] \\
ATC=E[Y^1(i,1)-Y^0(i,1)|R(i)=1] 
\end{eqnarray}


\begin{assumption}
Parallel trends in treatment and control: \\
$E[Y^0(i,1)-Y^0(i,0)|D=1,R=0]=E[Y^0(i,1)-Y^0(i,0)|D=0,R=0]$
\end{assumption}

\begin{assumption}
Parallel trends in close and control: \\
$E[Y^0(i,1)-Y^0(i,0)|D=0,R=1]=E[Y^0(i,1)-Y^0(i,0)|D=0,R=0]$
\end{assumption}

\begin{assumption}

\end{assumption}

\begin{assumption}
Violation of SUTVA are local: \\ 
\end{assumption}


\subsection{Parametric Estimation and Testing}
Formalise what is in the DPhil paper.

\subsection{Semi-Parametric Estimation of a Treatment Effect}
Robinson style back fit to get the coefficient on treatment (but not close).

\section{An Empirical Illustration: Spillovers and Contraceptive Reforms}
Two cases of contraceptive reforms.
\subsection{Abortion Reform in Mexico}
\subsubsection{Running Additional Placebo Tests}
Now there are at least two relevant placebo tests: pre reform between treat control
and pre-reform between close and non-close.

\subsection{Emergency Contraceptive Reform in Chile}


\section{Conclusion}

\bibliography{./biblio}

\end{spacing}
\end{document}
