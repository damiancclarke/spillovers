%********************************************************************************
\documentclass{article}


\usepackage{natbib}
\bibliographystyle{abbrvnat}
\bibpunct{(}{)}{;}{a}{,}{,}
\usepackage{setspace}
\usepackage{amsmath}

\setlength\parskip{0.25in}

\title{Estimating Difference-in-Differences in the Presence of Spillovers\footnote{
I thank participants in the Impact Evaluation Meeting at the Inter-American 
Development Bank for useful comments on this draft. Source code is available 
at https://github.com/damiancclarke/spillovers.  Affiliation:
Faculty of Economics, The University of Oxford, Manor Road, Oxford. Contact email: 
damian.clarke@economics.ox.ac.uk}}
\author{Damian Clarke}


\begin{document}

\maketitle

\begin{abstract}
I propose a method for difference-in-differences (DD) estimation in situations 
where the stable unit treatment value assumption is violated locally.  A flexible 
methodology is described to test for such spillovers, and to consistently estimate 
treatment effects in their presence.  This methodology is applied---both 
parametrically and semi-parametrically---to two empirical examples.
\end{abstract}

\newpage
\begin{spacing}{1.1}
%********************************************************************************
\section{Introduction}
Natural experiments often rely on territorial borders to estimate treatment 
effects.  These borders separate quasi-treatment from quasi-control groups with
individuals in one area having access to a program or treatment (hereafter 
program) while those in another do not.  In cases such as these where geographic 
location is used to motivate identification, the stable unit treatment value 
assumption (SUTVA) is, either explicitly or implicitly, invoked.

However, often territorial borders are porous.  Generally state, regional,
municipal, and village boundaries can be easily, if not costlessly, crossed.
Given this, researchers interested in using natural experiments in this way may
be concerned that the effects of a program in a treatment cluster may spillover 
into non-treatment clusters---at least locally.

Such a situation is in clear violation of the SUTVA's need that the treatment
status of any one unit must not affect the outcomes of any other unit.  In this 
paper we propose a simple methodology to deal with such spillover effects.  We
discuss how to test for local spillovers, and if such spillovers exist, how to 
estimate unbiased treatment effects.  It is shown that this estimation requires
a weaker condition than SUTVA: namely that SUTVA holds between \emph{some} units, 
as determined by their distance from the treatment cluster.  We show how to 
estimate treatment effects both parametrically and semi-parametrically, and then
propose a method to generalise the proposed estimator to a higher dimensional 
case where spillovers may depend, parametrically or non-parametrically, on a number 
of factors.

This methodology is then illustrated with 2 empirical examples.  We examine how
spillovers of reforms across municipal boundaries may contaminate `traditional' 
difference-in-differences (DD) estimators.  This is applied to two contraceptive
reforms, where individuals from contiguous or nearby areas can travel to a 
treatment region to access the reform.  It is shown that both the plausibly 
exogenous arrival of the morning after pill to certain municipalities of Chile, 
and abortion to certain districts in Mexico results in a reduction of births
in the given area, and in close-by quasi-control areas.  As a result, the
spillover-robust DD estimator propsed here flexibly captures this effect, 
correcting for any (local) spillover bias.



Ideas: \citet{McIntosh2008}, \citet{Bairdetal2014}, \citet{AngelucciDiMaro2010} 
Applications: \citet{AngelucciDeGiorgi2009}, \citet{Heckmanetal1998}, 
\citet{MiguelKremer2004}
Other: \citet{Imbens2004}, \citet{Heckmanetal1998b}, \citet{Manski2013}

%********************************************************************************
%SATURATE
\section{Methodology}
Define $Y(i,t)$ as the outcome for individual $i$ and time $t$.  The population
of interest is observed at 2 time periods, $t\in \{0,1\}$.  Assume that between
$t=0$ and $t=1$, some fraction of the population is exposed to a 
quasi-experimental treatment.  As per \citet{Abadie2005}, we will denote 
treatment for individual $i$ in time $t$ as $D(i,t)$, where $D(i,1)=1$ implies 
that the individual was treated, and $D(i,1)=0$ implies that the individual was
not directly treated.  Given that treatment only exists between periods 0 and 1,
$D(i,0)=0\ \forall\ i$.

It is shown by \citet{AshenfelterCard1985} that if the outcome is generated by
a component of variance process:
\begin{equation}
\label{Seqn:COV}
Y(i,t)=\delta(t) + \alpha D(i,t)+\eta(i)+\nu(i,t)
\end{equation}
where we follow \citet{Abadie2005} in using $\delta(t)$ to refer to a time-specific
component, $\alpha$ as the impact of treatment, $\eta(i)$ a component specific
to each individual, and $\nu(i,t)$ as a time-varying individual (mean zero) shock,
then a sufficient condition for identification is:
\begin{equation}
\label{Seqn:ID}
P(D(i,1)=1|\nu(i,t))=P(D(i,1)=1) \ \forall\ t\in\{0,1\}.
\end{equation}
In other words, identification requires that selection into treatment does not
rely on the unobserved time-varying component $\nu(i,t)$.  If this condition 
holds, then the classical DD estimator provides an unbiased estimate of the
treatment effect:
\begin{equation}
\label{Seqn:DD}
\begin{split}
\alpha&=\{E[Y(i,1)|D(i,1)=1]-E[Y(i,1)|D(i,1)=0]\} \\
      &-\{E[Y(i,0)|D(i,1)=1]-E[Y(i,0)|D(i,1)=0]\}.
\end{split}
\end{equation}

Assume now, however, that treatment is not precisely geographically bounded.  
Specifically, imagine that those living in control areas `close to' treatment 
areas themselves are able to access treatment, either partially or completely.
Such a case allows for a case where individuals `defy' their treatment status, 
by travelling or moving treated areas, or where spillovers from treatment
areas is diffused through general equilibrium processes.  Define $R(i,t)$ where
\begin{equation}
 R(i,t) =
  \begin{cases}
   1   & \text{if an individual lives close, but not in, a treatment area} \\
   0   & \text{otherwise} 
  \end{cases}
\end{equation}
As treatment occurs only in period 1, $R(i,0)=0$ for all $i$.  Similarly, as 
living in a treatment area itself excludes individuals from living `close to' 
the same treament area, $R(i,t)=0$ for all $i$ such that $D(i,t)=1$.

Generalising from (\ref{Seqn:COV}), now we assume that $Y(i,t)$ is generated 
by:
\begin{equation}
\label{Seqn:COV2}
Y(i,t)=\delta(t) + \alpha D(i,t)+\beta R(i,t)+\eta(i)+\nu(i,t)
\end{equation}
If we observe only $Y(i,t)$, $D(i,t)$ and $R(i,t)$, a sufficient condition for 
estimation now consists of (\ref{Seqn:ID}) and the following assumption: 
\begin{equation}
\label{Seqn:ID2}
P(R(i,1)=1|\nu(i,t))=P(R(i,1)=1) \ \forall\ t\in\{0,1\}.
\end{equation}
This says that both treatment, and being close to treatment cannot depend upon
individual-specific time-variant components. To see this, write (\ref{Seqn:COV2}), 
adding and subtracting $E[\eta(i)|D(i,1),
R(i,1)]$:
\begin{equation}
\label{Seqn:addsub}
Y(i,t)=\delta(t) + \alpha D(i,t)+\beta R(i,t)+E[\eta(i)|D(i,1),R(i,1)]+\varepsilon(i,t)
\end{equation}
where, following \citet{Abadie2005}, $\varepsilon(i,t)=\eta(i)-E[\eta(i)|D(i,1),R(i,1)]
+\nu(i,t)$.  We can redefine $\delta(t)=\delta(0)+[\delta(1)-\delta(0)]t$, and 
redefine $E[\eta(i)|D(i,1),R(i,1)]=E[\eta(i)|D(i,1)=0,R(i,1)=0]+(E[\eta(i)|D(i,1)=1,
R(i,1)=0]-E[\eta(i)|D(i,1)=0,R(i,1)=0])D(i,1)+(E[\eta(i)|D(i,1)=0,R(i,1)=1]-E[\eta(i)|
D(i,1)=0,R(i,1)=0])R(i,1)$.

\section{A Spillover-Robust Double Differences Estimator}
\subsection{Parametric Estimation and Testing}
Formalise what is in the DPhil paper.

\subsection{Semi-Parametric Estimation of a Treatment Effect}
Robinson style back fit to get the coefficient on treatment (but not close).

\section{An Empirical Illustration to Contraceptive Reforms}
Two cases of contraceptive reforms.
\subsection{Abortion Reform in Mexico}
\subsubsection{Running Additional Placebo Tests}
Now there are at least two relevant placebo tests: pre reform between treat control
and pre-reform between close and non-close.

\subsection{Emergency Contraceptive Reform in Chile}


\section{Conclusion}

\bibliography{./biblio}

\end{spacing}
\end{document}
