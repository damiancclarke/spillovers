%********************************************************************************
\documentclass{article}


\usepackage{natbib}
\bibliographystyle{abbrvnat}
\bibpunct{(}{)}{;}{a}{,}{,}
\usepackage{setspace}

\setlength\parskip{0.25in}

\title{Estimating Treatment Effects in the Presence of Spillovers\footnote{
I thank Christopher Roth and participants in the Impact Evaluation Meeting
at the Inter-American Development Bank for useful comments for this draft.
Source code is available at \texttt{url}.}}
\author{Damian Clarke\thanks{\texttt{damian.clarke@economics.ox.ac.uk}}}


\begin{document}

\maketitle

\begin{abstract}
We propose a method to estimate treatment effects in situations where the stable
unit treatment value assumption is violated locally.  A flexible methodology is
described to test for such spillovers, and to consistently estimate treatment 
effects in their presence.  This methodology is applied---both parametrically and
semi-parametrically---to an empirical example.
\end{abstract}

%********************************************************************************
\section{Introduction}
Natural experiments often rely on territorial borders to estimate treatment 
effects.  These borders separate quasi-treatment from quasi-control groups with
individuals in one area having access to a program\footnote{In this paper we will
refer to the `treatment' as a program.  However, the results here hold without 
loss of generality for other economic phenomena which are not necessarily 
explicit programs (such as, for example, a change in the prevailing minimum wage)} 
while those in another do not.  In cases such as these where geographic location 
is used to motivate identification, the stable unit treatment value assumption 
(SUTVA) is, either explicitly or implicitly, invoked.

However, often territorial borders are porous.  Generally state, regional,
municipal, and village boundaties can be easily, if not costlessly, crossed.
Given this, researchers interested in using natural experiments in this way may
be concerned that the effects of a program in a treatment cluster may spillover 
into non-treatment clusters---at least locally.

Such a situation is in clear violation of the SUTVA's need that the treatment
status of any one unit must not affect the outcomes of any other unit.  In this 
paper we propose a simple methodology to deal with such spillover effects.  We
discuss how to test for local spillovers, and if such spillovers exist, how to 
estimate unbiased treatment effects.  It is shown that this estimation requires
a weaker condition than SUTVA: namely that SUTVA holds between \emph{some} units, 
as determined by their distance from the treatment cluster.  We show how to 
estimate treatment effects both parametrically and semi-parametrically, and then
briefly apply this to an empirical example.


Ideas: \citet{McIntosh2008}, \citet{Bairdetal2014}, \citet{AngelucciDiMaro2010} 
Applications: \citet{AngelucciDeGiorgi2009}, \citet{Heckmanetal1998}, 
\citet{MiguelKremer2004}
Other: \citet{Imbens2004}

%********************************************************************************
%SATURATE
\section{Methodology}
Consider an outcome variable $y$, a binary treatment variable $T$, and a measure 
$d$ referring to the distance between each unit of observation $i$ and its 
nearest treatment cluster $c$.  In this general formulation we will assume (for 
notational simplicity that the treatment effect can be identified using single
differences, however in the empirical illustration in the following section
will also show its application to a difference-in-differences setup.

\section{A Spillover-Robust Double Differences Estimator}
\subsection{Parametric Estimation and Testing}
Formalise what is in the DPhil paper.

\subsection{Semi-Parametric Estimation of a Treatment Effect}
Robinson style back fit to get the coefficient on treatment (but not close).

\section{An Empirical Illustration}
\section{Conclusion}

\bibliography{./biblio}

\end{document}
